

\pdfoutput=1

\documentclass{llncs}

\usepackage[utf8]{inputenc}

\usepackage[protrusion=true,expansion=true]{microtype}

\usepackage[english]{babel}

\usepackage[style=numeric,
%  backref=true,
 isbn=false,
 maxnames=3,
 maxbibnames=99 ,                
 uniquename=init ,
]{biblatex}
\bibliography{literature.bib}

\usepackage{ownstylesansthm}



\newcommand{\fat}[1]{\textbf{#1}}


\begin{document}

\title{Terminal semantics for codata types\\in intensional Martin-Löf type theory}

\author{Benedikt Ahrens and R\'egis Spadotti}

\institute{
Institut de Recherche en Informatique de Toulouse\\
Universit\'e Paul Sabatier, 
Toulouse}



\maketitle

% \tableofcontents


In this work, we study the notions of \emph{relative comonad} and \emph{comodule over a relative comonad}.
 We then use these notions to characterise several \emph{co}inductive data types in intensional Martin-L\"of type theory
 via a universal property.
 
 In a set-theoretic setting, inductive sets are characterized as initial algebras for 
 some endofunctor on the category of sets. For instance, the set of natural numbers constitutes the carrier of the 
 initial algebra of the functor $X \mapsto 1 + X$.
 
 In a type-theoretic setting as given by Martin-L\"of type theory \parencite{martin_lof}, 
 two approaches to the semantics of inductive types have been studied: 
 one approach consists in showing that inductive types exist in a \emph{model} of the type theory,
 as is done by \textcite{DBLP:journals/apal/MoerdijkP00}.
 Another approach is to prove that adding certain type-theoretic rules to the type theory
 implies (or is equivalent to) the existence of a universal object \emph{within} type theory
 (see, e.g., \parencite{DBLP:conf/lics/AwodeyGS12, DBLP:journals/tcs/Dybjer97}).
 This latter approach is the one we take in the present work.
 

 Dually to inductive sets, \fat{co}inductive sets such as streams are characterised as terminal objects \parencite{jacobs1997tutorial}.
 Inhabitants of such sets are equal if and only if they are \emph{bisimilar} \parencite{DBLP:journals/mscs/TuriR98}:
 Intuitively, two elements of a coinductive set are the same if they allow for the same observations.
 
 This correspondence between equality and bisimilarity fails in IMLTT, when equality is considered to be given by the 
 Martin-L\"of identity type. Instead, one defines \emph{bisimilarity} as a coinductive predicate 
 on a coinductive type, and one reasons about the terms of a coinductive type modulo the bisimilarity predicate
 rather than identity \parencite{DBLP:conf/types/Coquand93}.
 Consequently, we consider two maps into a coinductive type to be the same if they are \emph{pointwise bisimilar}---an analogue
 to the aforementioned principle of function extensionality. 
 With these conventions, we give, in the present work, a characterisation of some coinductive data types as \emph{terminal} objects in some category 
 defined in intensional Martin-L\"of type theory.
 More precisely, we consider an example of \emph{homogeneous} codata type, streams, and 
 an example of \emph{heterogeneous} codata type, triangular matrices.
 For each of these examples we prove, 
 from type-theoretic rules specifying the respective codata type added to the basic rules of Martin-L\"of type theory,
 the existence of a terminal object in some category \emph{within IMLTT}.
 Our terminal semantics characterises not only the codata types themselves but also the bisimilarity relation and
 a canonical cosubstitution operation on them.
 
 The fact that cosubstitution for coinductive data types is comonadic in a set-theoretic setting is established by \textcite{DBLP:conf/sfp/UustaluV01}.
 In IMLTT however, in order to characterize that cosubstitution operation on a given codata type, and its algebraic properties,
 we develop the notion of \emph{relative comonad} and \emph{comodule over a relative comonad}.
 The need to consider \emph{relative} comonads arises from the  need to check the algebraic properties of cosubstitution modulo \emph{bisimilarity} rather
 than modulo identity (in the sense of ML identity types).
 
 All our results have been implemented in the proof assistant \coq \parencite{coq84pl3}.
 The \coq source files and HTML documentation are available on the project web page \parencite{trimat_coq}.
 
\renewcommand*{\bibfont}{\small}
\printbibliography[heading=none]



\end{document}



















