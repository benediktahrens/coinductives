

\pdfoutput=1

\documentclass{llncs}

\usepackage[utf8]{inputenc}

\usepackage[protrusion=true,expansion=true]{microtype}


\usepackage[style=numeric,
 backref=true,
 isbn=false,
 maxnames=3,
 maxbibnames=99 ,                
 uniquename=init ,
]{biblatex}
\bibliography{literature.bib}

\usepackage{ownstylesansthm}



\newcommand{\fat}[1]{\textbf{#1}}


\begin{document}

\title{Coinitial semantics \\ for redecoration of triangular matrices}

\author{Benedikt Ahrens and R\'egis Spadotti}

\institute{
Institut de Recherche en Informatique de Toulouse\\
Universit\'e Paul Sabatier\\
Toulouse}


% \begin{abstract}
%   The heterogeneous codata type of infinite triangular matrices and, in particular, the \emph{redecoration} operation on it, 
%   were studied by \citeauthor{DBLP:conf/types/MatthesP11}. In their work,  redecoration is characterized
%   as the cobind operation of what the authors call a \enquote{weak constructive comonad}.
%   
%   In this work, we identify weak constructive comonads as an instance of the more general notion of 
%   \emph{relative comonad}.
%   Afterwards, building upon the work by \citeauthor{DBLP:conf/types/MatthesP11}, we give a category-theoretic
%   characterization of infinite triangular matrices---equipped with the canonical bisimulation relation and a 
%   compatible comonadic $\cobind$ operation---as the \emph{terminal object}
%   in some category.
%   
%   Our results are fully mechanized in the proof assistant \coq.
%   \end{abstract}

\maketitle

% \tableofcontents



 Simple inductive types---\textsf{W}-types---are characterized categorically as initial algebras of
 a polynomial functor.
 Dually, infinite, \emph{coinductive} types are characterized as terminal \emph{co}algebras of suitable functors.
 In the case of coinductive types, the meta-theoretic notion of equality is not adequate: instead, the notion of 
 \emph{bisimulation}, was introduced by \textcite{aczel_nonwellfounded}.

 
 The characterization of inductive types as initial algebras 
 has been extended to \emph{heterogeneous}---also called \emph{nested}---inductive data types, e.g., the type of $\lambda$-terms,
 in various formulations \parencite{fpt, DBLP:journals/iandc/HirschowitzM10}.
 The main goal of these works is not just to characterize the data type via a universal property, but rather the data type
 \emph{equipped with a well-behaved substitution operation}.
 
 

 In this work we study a specific \emph{co}inductive heterogeneous data type---the type family $\Tri$ of 
 infinite triangular matrices also studied by \textcite{DBLP:conf/types/MatthesP11}, and its \emph{redecoration} operation.
 The codata type of infinite triangular matrices and the \emph{redecoration} operation on it was analyzed by 
 \textcite{DBLP:conf/types/MatthesP11}: the codata type family $\Tri$ of infinite triangular matrices is parametrized by a fixed set $E$ for entries not on the diagonal, 
 and indexed by another, \emph{variable}, set $A$ for entries on 
 the diagonal. The respective types of its specifying destructors $\head$ and $\tail$ are given in \Cref{fig:tri_destructors},
 together with the generating rules for the \fat{bisimulation} relation on it.
 
 More precisely, we characterize this codata type, equipped with \emph{co}substitution operation, as a terminal object of some category.
 For this, we dualize the approach by \textcite{DBLP:journals/iandc/HirschowitzM10}, where 
 the crucial notions are the notion of monad and, more importantly, \emph{module over a monad}.
 It turns out that more work than a simple dualization is necessary, for two reasons:
 \begin{itemize}
  \item the lambda calculus can be seen as a monad on sets and thus, in particular, as an endofunctor.
        The codata type $\Tri$, however, associates to any \emph{set} of potential diagonal elements a \emph{setoid}
        of triangular matrices. We thus need a notion of comonad whose underlying functor is not necessarily endo.
  \item the category-theoretic analysis of the destructor $\tail$ is more complicated than the heterogeneous 
           constructor of abstraction of the lambda calculus.
 \end{itemize}

 The first of these issues is easily fixed by considering \emph{relative} comonads, the notion dual to the
 relative monads introduced by
  \parencite{DBLP:conf/fossacs/AltenkirchCU10}. 
  
 The second issue necessitates the introduction of the notion of \emph{relative comonad with cut}, that is,
 a relative comonad equipped with an extra operation called $\cut$ that satisfies some compatibility rules.
 
 
 
 
 The other, more important difference between the inductive and the coinductive case stems from \emph{heterogeneity}:
 Two dual operations are responsible for the heterogeneity of the data type of lambda terms and the codata type of infinite triangular
 matrices: while we precompose with taking a \emph{co}product in the case of the abstraction of the lambda calculus---in order to account
 for context extension when passing under a binder---,
 we precompose with taking a \emph{product} when giving the target of the $\tail$ destructor of $\Tri$.
 However, as it turns out, the treatment of the $\tail$ destructor is far more complicated than a dualization of that of the abstraction 
 constructor of the lambda calculus---see \Cref{rem:shift} for a detailed explanation. 
 Describing our approach to a category-theoretic treatment of the $\tail$ destructor is the main purpose of the present work.
 

 
 

\begin{figure}[hbt]
  \begin{center}

     \def\extraVskip{3pt}
     \def\proofSkipAmount{\vskip.8ex plus.8ex minus.4ex}
    \AxiomC{$t : \Tri(A)$}\doubleLine
     \UnaryInfC{$\head_A(t) : A$}
      \DisplayProof
                        \hspace{3ex}
                                       \AxiomC{$t : \Tri(A)$}\doubleLine
                                       \UnaryInfC{$\tail_A(t) : \Tri(E\times A)$}
                                       \DisplayProof%

  \end{center}
\vspace{1ex}
  \begin{center}
                                            \def\extraVskip{3pt}
     \def\proofSkipAmount{\vskip.8ex plus.8ex minus.4ex}
    \AxiomC{$t \sim t'$}\doubleLine
     \UnaryInfC{$\head(t) = \head(t')$}
      \DisplayProof
                        \hspace{3ex}
                                       \AxiomC{$t \sim t'$}\doubleLine
                                       \UnaryInfC{$ \tail(t) \sim \tail(t')$}
                                       \DisplayProof   
  \end{center}
  \caption{Destructors and bisimulation for the coinductive family of setoids $\Tri$} \label{fig:tri_destructors}
\end{figure}


The notion of relative comonad captures many properties of $\Tri$ and its redecoration operation, in particular the interplay
of redecoration with the destructor $\head$ via the first two comonad laws.
Much of the rest of the present work is concerned with finding a suitable categorical notion that captures the interplay 
of redecoration with the other destructor, $\tail$. Once we have found such a categorical notion, we can use 
it to give a definition of what a \enquote{coalgebra} for the signature of infinite triangular matrices is, 
where the codata type $\Tri$ is supposed to constitute the terminal object.





For heterogeneous inductive types, \textcite{DBLP:journals/iandc/HirschowitzM10} solve an analogous question by considering the notion of
\emph{module over a monad}. We  dualize their approach by introducing comodules over (relative) comonads.






\section{Coalgebras for infinite triangular matrices}\label{sec:coalgebras_for_tri}


Our goal is to define \fat{coalgebras} for the signature of infinite triangular matrices such that 
the codata type $\Tri$ defined in \Cref{ex:tri_comonad} constitutes the \fat{terminal such coalgebra}.
Since $\Tri$ forms a comonad relative to the functor $\eq:\Set\to\Setoid$, it seems reasonable to define
an arbitrary coalgebra for that signature as such a relative comonad as well, equipped with some extra structure which remains to be defined.
A morphism of coalgebras would then be defined as a morphism of relative comonads that is compatible with that extra structure.

The purpose of this extra structure is to characterize the destructor $\tail$, that is, its \emph{type} as well as
its \emph{behaviour with respect to the redecoration operation}.
Dualizing the approach of \textcite{DBLP:journals/iandc/HirschowitzM10}, we hope to characterize $\tail$ as a morphism of 
comodules of specified type, which would allow us to define the extra structure of an arbitrary coalgebra for that signature
as a suitable comodule morphism as well.

In order to turn $\tail$ into a morphism of comodules, the crucial task is to suitably endow the source and target functors
$\Tri$ and $\Tri(E\times \_)$, respectively, with a comodule structure over the comonad $\Tri$.
For this, we now take a closer look at how the $\extend$ operation for $\Tri$ is defined. Its purpose is to 
modify the function $f$ used for redecoration, when redecorating $\tail~t : \Tri(E\times A)$ rather than $t : \Tri A$.
First we note an important difference between the $\extend$ operation and its analogue in the inductive case:



The failure to describe the $\extend$ operation---more precisely, the auxiliary $\cut$---in categorical terms leads us 
to considering comonads that come with a specified $\cut$ operation for some fixed object $E$ in the source category:














Given a comodule $M$ over a relative comonad with cut, we define a comodule over the same comonad obtained by precomposition of $M$ with
\enquote{product with a fixed object $E$}:



\begin{example}[$\tail$ is a comodule morphism]\label{ex:tail_comodule}
 Consider the comonad $\Tri$, equipped with the $\cut$ operation of \Cref{def:cut_for_tri}.
 The destructor \constfont{\tail} of \Cref{ex:tri_comonad} is a morphism of comodules over the comonad $\Tri$ 
  from the tautological comodule  $\Tri$ to $\Tri(E\times \_)$.
\end{example}








We now define a category in which the codata type of triangular matrices, equipped with the bisimulation relation of \Cref{ex:tri_comonad}
and the comonadic redecoration, constitutes the terminal object. We call an object of this category a \emph{coalgebra of triangular matrices}, 
even though we do not define the category as a category of coalgebras of a given endofunctor.

The definition of such a coalgebra is obtained by collecting all the information we have gathered so
far about the comonad $\Tri$, and abstracting from this particular comonad:


\begin{definition}[Coalgebras of infinite triangular matrices]\label{def:cat_tri}
   Let $E:\Set_0$ be a set.
   Let $\mathcal{T} = \mathcal{T}_E$ be the category where an object consists of
   \begin{itemize}
    \item a comonad $T$ over the functor $\eq:\Set\to\Setoid$ with $\cut$ relative to $E$ and
    \item a morphism $\tail$ of comodules over $T$ of type $T \to T(E\times \_)$
   \end{itemize}
   such that for any set $A$,
    \[ \comp{\cut_A}{\tail_A} = \comp{\tail_{E\times A}}{\cut_{E\times A}} \enspace . \]
   The last equation can be stated as an equality of comodule morphisms as
     \[ \comp{\cut}{\tail} = \comp{\tail(E\times \_)}{\cut(E\times\_)} \quad \bigl( = (\comp{\tail}{\cut})(E\times \_)\bigr)\enspace . \]

  
   
   A morphism between two such objects $(T,\tail^T)$ and $(S,\tail^S)$
   is given by a morphism of relative comonads with cut $\tau : T \to S$ such that
   the following diagram of comodule morphisms in the category $\RComod(S,\E)$ commutes,
   

   \noindent
   Here in the upper right corner we silently insert an isomorphism as in \Cref{rem:prod_pullback_commute}.
\end{definition}   
   
We now have all the ingredients to state (and prove) our main theorem:
\begin{theorem}[Coinitial semantics for triangular matrices with redecoration]\label{ex:final_sem_tri} % checked   
   The pair $(\Tri, \tail)$ consisting of the relative comonad with cut $\Tri$ of \Cref{def:cut_for_tri} together with 
    the morphism of comodules $\tail$ of \Cref{ex:tail_comodule},
   constitutes the terminal---\enquote{coinitial}---coalgebra of triangular matrices.
\end{theorem}


This universal property of coinitiality characterizes not only the codata type of infinite triangular matrices, but also
the \fat{bisimulation} relation on it as well as the \fat{redecoration} operation.


 All our definitions, examples and lemmas have been implemented in the proof assistant \coq \parencite{coq84pl3}.
 The \coq source files and HTML documentation are available on the project web page \parencite{trimat_coq}.



While a significant part of our work---surmounting the non-categoricity of the $\cut$ operation---seems to be specific to this particular codata type,
we believe that our work proves the suitability of the notion of relative (co)monads and (co)modules thereover for 
a categorical treatment of coinductive data types.


We plan to pursue two obvious lines of work:
Firstly, we will work on a suitable notion of \emph{signature} for the specification of coinductive data types.
Secondly, we hope to integrate \emph{equations} into the notion of signature, equations which 
will allow, e.g., considering branching trees modulo permutation of subtrees.
 



\printbibliography



\end{document}



















