

\pdfoutput=1

\documentclass{llncs}

\usepackage[utf8]{inputenc}

\usepackage[protrusion=true,expansion=true]{microtype}


\usepackage[style=numeric,
%  backref=true,
 isbn=false,
 maxnames=3,
 maxbibnames=99 ,                
 uniquename=init ,
]{biblatex}
\bibliography{literature.bib}

\usepackage{ownstylesansthm}



\newcommand{\fat}[1]{\textbf{#1}}


\begin{document}

\title{Coinitial semantics \\ for redecoration of triangular matrices}

\author{Benedikt Ahrens and R\'egis Spadotti}

\institute{
Institut de Recherche en Informatique de Toulouse\\
Universit\'e Paul Sabatier, 
Toulouse}


% \begin{abstract}
%   The heterogeneous codata type of infinite triangular matrices and, in particular, the \emph{redecoration} operation on it, 
%   were studied by \citeauthor{DBLP:conf/types/MatthesP11}. In their work,  redecoration is characterized
%   as the cobind operation of what the authors call a \enquote{weak constructive comonad}.
%   
%   In this work, we identify weak constructive comonads as an instance of the more general notion of 
%   \emph{relative comonad}.
%   Afterwards, building upon the work by \citeauthor{DBLP:conf/types/MatthesP11}, we give a category-theoretic
%   characterization of infinite triangular matrices---equipped with the canonical bisimulation relation and a 
%   compatible comonadic $\cobind$ operation---as the \emph{terminal object}
%   in some category.
%   
%   Our results are fully mechanized in the proof assistant \coq.
%   \end{abstract}

\maketitle

% \tableofcontents



 Simple inductive types---\textsf{W}-types---are characterized categorically as initial algebras of
 a polynomial functor.
 Dually, \emph{co}inductive types are characterized as terminal \emph{co}algebras of polynomial functors.
 In the case of coinductive types, the meta-theoretic notion of equality is not adequate: thus, the notion of 
 \emph{bisimulation} was introduced by \textcite{aczel_nonwellfounded}.

 
 The characterization of inductive types as initial algebras 
 has been extended to \emph{heterogeneous}---also called \emph{nested}---inductive data types, e.g., the type of $\lambda$-terms,
 in various formulations \parencite{fpt, DBLP:journals/iandc/HirschowitzM10}.
 The main goal of those works is to characterize not only the data type via a universal property, but rather the data type
 \emph{equipped with a well-behaved substitution operation}.
 
 

 In the present work we study a specific \emph{co}inductive \emph{heterogeneous} data type---the type family $\Tri$ of 
 infinite triangular matrices---and its \emph{redecoration} operation:
 the codata type is parametrized by a fixed set $E$ for entries not on the diagonal, 
 and indexed by another, \emph{variable}, set $A$ for entries on 
 the diagonal. The respective types of its specifying destructors $\head$ and $\tail$ are given in \Cref{fig:tri_destructors},
 together with the destructors for the coinductively defined bisimilarity relation on it.
 Equipped with the redecoration operation, the type $\Tri$ is shown by \textcite{DBLP:conf/types/MatthesP11}
 to constitute what they call a \enquote{weak constructive comonad}.
 \begin{figure}[bt]
  \begin{center}

     \def\extraVskip{3pt}
     \def\proofSkipAmount{\vskip.8ex plus.8ex minus.4ex}
    \AxiomC{$t : \Tri(A)$}\doubleLine
     \UnaryInfC{$\head_A(t) : A$}
      \DisplayProof
                        \hspace{3ex}
                                       \AxiomC{$t : \Tri(A)$}\doubleLine
                                       \UnaryInfC{$\tail_A(t) : \Tri(E\times A)$}
                                       \DisplayProof%

   \end{center}
%  \vspace{1ex}
   \begin{center}
                                            \def\extraVskip{3pt}
     \def\proofSkipAmount{\vskip.8ex plus.8ex minus.4ex}
    \AxiomC{$t \sim t'$}\doubleLine
     \UnaryInfC{$\head(t) = \head(t')$}
      \DisplayProof
                        \hspace{3ex}
                                       \AxiomC{$t \sim t'$}\doubleLine
                                       \UnaryInfC{$ \tail(t) \sim \tail(t')$}
                                       \DisplayProof   
  \end{center}
  \caption{Destructors and bisimilarity for the coinductive family of setoids $\Tri$} \label{fig:tri_destructors}
\end{figure}

 
 In this work, we first identify those weak constructive comonads as an instance of the more general notion of \emph{relative comonad}.
 Indeed, a weak constructive comonad is precisely a comonad relative to the functor $\eq:\Set\to\Setoid$ from the category of sets to that of setoids that is 
 left adjoint to the forgetful functor.
 
 Afterwards, we characterize the codata type $\Tri$, equipped with the cosubstitution operation of redecoration, as a terminal object of some category.
 For this, we dualize the approach by \textcite{DBLP:journals/iandc/HirschowitzM10},
 who characterize the heterogeneous inductive type of lambda terms---equipped with a suitable substitution operation---as an initial object in
 a category of algebras for the signature of lambda terms.
 In their work, the crucial notions are the notion of monad and, more importantly, \emph{module over a monad}.
 It turns out that more work than a simple dualization is necessary for two reasons:
 \begin{itemize}
  \item the lambda calculus can be seen as a monad on sets and thus, in particular, as an endofunctor.
        The codata type $\Tri$, however, associates to any \emph{set} of potential diagonal elements a \emph{setoid}
        of triangular matrices. We thus need a notion of comonad whose underlying functor is not necessarily endo: the 
        already mentioned \emph{relative} comonads;
  \item the category-theoretic analysis of the destructor $\tail$ is more complicated than that of the heterogeneous 
           constructor of abstraction of the lambda calculus.
 \end{itemize}
 Finding a suitable categorical notion to capture the destructor $\tail$ and, more importantly, its interplay with
 the comonadic redecoration operation on $\Tri$, constitutes the main contribution of the present work.
 These rather technical details shall not be explained in this extended abstract.
 
 



  Once we have found such a categorical notion, we can use 
  it to give a definition of a \enquote{coalgebra} for the signature of infinite triangular matrices, 
  together with a suitable notion of \emph{morphism} of such coalgebras.
   We thus obtain a category of coalgebras for that signature.
  Any object of this category comes with a comonad relative to the aforementioned functor $\eq : \Set\to\Setoid$
  and a suitable comodule over this comonad, modeling in some sense the destructor $\tail$.
  Our main result then states that this category 
  has a terminal object built from the codata type $\Tri$ and its destructor $\tail$,
  which are seen as a relative comonad and a comodule over that relative comonad, respectively.
% 
% 
% 
% 
% \section{Coalgebras for infinite triangular matrices}\label{sec:coalgebras_for_tri}
% 
% 
% Our goal is to define \fat{coalgebras} for the signature of infinite triangular matrices such that 
% the codata type $\Tri$ defined in \Cref{ex:tri_comonad} constitutes the \fat{terminal such coalgebra}.
% Since $\Tri$ forms a comonad relative to the functor $\eq:\Set\to\Setoid$, it seems reasonable to define
% an arbitrary coalgebra for that signature as such a relative comonad as well, equipped with some extra structure which remains to be defined.
% A morphism of coalgebras would then be defined as a morphism of relative comonads that is compatible with that extra structure.
% 
% The purpose of this extra structure is to characterize the destructor $\tail$, that is, its \emph{type} as well as
% its \emph{behaviour with respect to the redecoration operation}.
% Dualizing the approach of \textcite{DBLP:journals/iandc/HirschowitzM10}, we hope to characterize $\tail$ as a morphism of 
% comodules of specified type, which would allow us to define the extra structure of an arbitrary coalgebra for that signature
% as a suitable comodule morphism as well.
% 
% In order to turn $\tail$ into a morphism of comodules, the crucial task is to suitably endow the source and target functors
% $\Tri$ and $\Tri(E\times \_)$, respectively, with a comodule structure over the comonad $\Tri$.
% For this, we now take a closer look at how the $\extend$ operation for $\Tri$ is defined. Its purpose is to 
% modify the function $f$ used for redecoration, when redecorating $\tail~t : \Tri(E\times A)$ rather than $t : \Tri A$.
% First we note an important difference between the $\extend$ operation and its analogue in the inductive case:
% 
% 
% 
% The failure to describe the $\extend$ operation---more precisely, the auxiliary $\cut$---in categorical terms leads us 
% to considering comonads that come with a specified $\cut$ operation for some fixed object $E$ in the source category:
% 
% 
% 
% 
% 
% 
% 
% 
% 
% 
% 
% We now define a category in which the codata type of triangular matrices, equipped with the bisimulation relation of \Cref{ex:tri_comonad}
% and the comonadic redecoration, constitutes the terminal object. We call an object of this category a \emph{coalgebra of triangular matrices}, 
% even though we do not define the category as a category of coalgebras of a given endofunctor.
% 
% The definition of such a coalgebra is obtained by collecting all the information we have gathered so
% far about the comonad $\Tri$, and abstracting from this particular comonad:
% 
% 
% \begin{definition}[Coalgebras of infinite triangular matrices]\label{def:cat_tri}
%    Let $E:\Set_0$ be a set.
%    Let $\mathcal{T} = \mathcal{T}_E$ be the category where an object consists of
%    \begin{itemize}
%     \item a comonad $T$ over the functor $\eq:\Set\to\Setoid$ with $\cut$ relative to $E$ and
%     \item a morphism $\tail$ of comodules over $T$ of type $T \to T(E\times \_)$
%    \end{itemize}
%    such that for any set $A$,
%     \[ \comp{\cut_A}{\tail_A} = \comp{\tail_{E\times A}}{\cut_{E\times A}} \enspace . \]
%    The last equation can be stated as an equality of comodule morphisms as
%      \[ \comp{\cut}{\tail} = \comp{\tail(E\times \_)}{\cut(E\times\_)} \quad \bigl( = (\comp{\tail}{\cut})(E\times \_)\bigr)\enspace . \]
% 
%   
%    
%    A morphism between two such objects $(T,\tail^T)$ and $(S,\tail^S)$
%    is given by a morphism of relative comonads with cut $\tau : T \to S$ such that
%    the following diagram of comodule morphisms in the category $\RComod(S,\E)$ commutes,
%    
% 
%    \noindent
%    Here in the upper right corner we silently insert an isomorphism as in \Cref{rem:prod_pullback_commute}.
% \end{definition}   
%    
% We now have all the ingredients to state (and prove) our main theorem:
% \begin{theorem}[Coinitial semantics for triangular matrices with redecoration]\label{ex:final_sem_tri} % checked   
%    The pair $(\Tri, \tail)$ consisting of the relative comonad with cut $\Tri$ of \Cref{def:cut_for_tri} together with 
%     the morphism of comodules $\tail$ of \Cref{ex:tail_comodule},
%    constitutes the terminal---\enquote{coinitial}---coalgebra of triangular matrices.
% \end{theorem}
% 
% 
This universal property of coinitiality characterizes not only the codata type of infinite triangular matrices but also
the \fat{bisimilarity} relation on it as well as the \fat{redecoration} operation.
 
 All our definitions, examples, and lemmas have been implemented in the proof assistant \coq. % \parencite{coq84pl3}.
 The \coq source files and HTML documentation are available on the project web page \parencite{trimat_coq}.
 While parts of our work seem to be specific to the particular codata type $\Tri$,
 we believe that our work proves the suitability of the notion of relative (co)monads and (co)modules thereover for 
 a categorical characterization of coinductive data types.


% We plan to pursue two obvious lines of work:
% Firstly, we will work on a suitable notion of \emph{signature} for the specification of coinductive data types.
% Secondly, we hope to integrate \emph{equations} into the notion of signature, equations which 
% will allow, e.g., considering branching trees modulo permutation of subtrees.
 


\renewcommand*{\bibfont}{\small}
\printbibliography[heading=none]



\end{document}



















