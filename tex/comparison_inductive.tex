


\section{$\cut$ is not a redecoration}

\begin{remark}
 $\cut$ is not a redecoration. For suppose we have that for some $f : \Tri(E\times A) \to \Tri A$ and for any matrix $t$, $\redec~f~t \sim \cut~t$.
 By the rule for $\head$ of \Cref{fig:tri_destructors} it follows that for any $t$ we have 
 $f~t = \comp{\head}{\pr_2}~t$.
 If we then compute 
 
  \[ \cut~t = \cut 
              \begin{pmatrix}
                     e_{1} & e_{2} & \cdots   \\
                     a_{1} & e_{3} & \cdots   \\
                           & a_{2} & \cdots   \\
                           &       & \vdots  
              \end{pmatrix} =              
              \begin{pmatrix}
%                      e_{1} & e_{2} & \cdots   \\
                     a_{1} & e_{3} & \cdots   \\
                           & a_{2} & \cdots   \\
                           &       & \vdots  
              \end{pmatrix}              
  \]
 while for the redecoration with $f = \comp{\head}{\pr_2}$ we obtain
  \[ \redec~f~ 
              \begin{pmatrix}
                     e_{1} & e_{2} & \cdots   \\
                     a_{1} & e_{3} & \cdots   \\
                           & a_{2} & \cdots   \\
                           &       & \vdots  
              \end{pmatrix} =              
              \begin{pmatrix}
%                      e_{1} & e_{2} & \cdots   \\
                     a_{1} & e_{2} & \cdots   \\
                           & a_{2} & \cdots   \\
                           &       & \vdots  
              \end{pmatrix}              
  \]
  meaning that $\cut~t \neq \redec~f~t$ as soon as $e_2 \neq e_3$.
\end{remark}



% \section{Some remarks about the difference between lambda calculus and triangular matrices}\label{sec_comp}
% 
% 
% \begin{definition}
%  Define the inductive 
% \end{definition}
