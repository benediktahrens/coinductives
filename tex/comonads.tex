

\pdfoutput=1

\documentclass{amsart}

\usepackage[protrusion=true,expansion=true]{microtype}


\usepackage[style=numeric,
 backref=true,
 isbn=false,
 maxnames=3,
 maxbibnames=99 ,                
 uniquename=init ,
]{biblatex}
\bibliography{literature.bib}

\usepackage{ownstyle}


\author{Benedikt Ahrens}
\author{R\'egis Spadotti}

\title[Coinitial semantics for redecoration of triangular matrices]{Coinitial semantics \\ for redecoration of triangular matrices}
% Comodules over relative comonads for coinitial semantics
\newcommand{\fat}[1]{\textbf{#1}}

\begin{document}


\begin{abstract}
  Infinite triangular matrices and, in particular, the \emph{redecoration} operation on them, 
  were studied by \citeauthor{DBLP:conf/types/MatthesP11}. In their work,  redecoration is characterized
  as the cobind operation of what the authors call a \enquote{weak constructive comonad}.
  
  In this work, we identify weak constructive comonads as an instance of the more general notion of 
  \emph{relative comonad}.
  Afterwards, building upon the work by \citeauthor{DBLP:conf/types/MatthesP11}, we give a category-theoretic
  characterisation of infinite triangular matrices---equiped with the canonical bisimulation relation and a 
  compatible comonadic $\cobind$ operation---as the \emph{terminal object}
  in some category.
  
%   We give a category-theoretic characterisation of heterogeneous coinductive datatypes as terminal---\enquote{coinitial}---objects.
% 
%   For this, we develop the notion of \emph{comodule over a relative comonad} and provide some constructions of such comodules.

%   The right notion of sameness for terms of a coinductive datatype is not equality but \emph{bisimulation}.
%   Consequently, we model coinductive datatypes that are parametrized by a set not as a comonad on the category of sets,
%   but as a \emph{relative} comonad from sets to setoids, over a suitable functor.
\end{abstract}

\maketitle

% \tableofcontents

\section{Introduction}

 Simple inductive types---\textsf{W}-types---are characterised categorically as initial algebras of
 a polynomial functor \parencite{DBLP:journals/apal/MoerdijkP00}.
 Dually, infinite, \emph{coinductive} types are characterised as terminal \emph{co}algebras of suitable functors.
 In the case of coinductive types, the meta-theoretic notion of equality is not adequate. Another notion of sameness for inhabitants of such types, 
 \emph{bisimulation}, was introduced by \textcite{aczel_nonwellfounded}, and also
 a new proof principle called \emph{coinduction}. 
 Modulo this slight complication, it looks like it suffices to dualize the theory of inductive types in order to obtain
 a complete picture of coinductive types.
%  It turns out that the story is not that easy when considering \emph{heterogeneous} (co)inductive types:
 
 The characterisation of inductive types as initial algebras 
 has been extended to \emph{heterogeneous}---also called \emph{nested}---inductive datatypes, e.g., the type of $\lambda$-terms,
 in various formulations \parencite{fpt, DBLP:journals/iandc/HirschowitzM10}.
 The main goal of these works is not just to characterize the data type via a universal property, but rather the data type
 \emph{equiped with a well-behaved substitution operation}.
 
 
%  Such coinductive datatypes are characterised as terminal---coinitial---algebras of polynomial functors.

 In this work we set off to characterize a specific \emph{co}inductive heterogeneous datatype---the type family $\Tri$ of 
 infinite triangular matrices as studied by \textcite{DBLP:conf/types/MatthesP11}, equiped with a \emph{co}substitution operation---as a terminal object of some 
 category. For this, we start dualizing the approach by \textcite{DBLP:journals/iandc/HirschowitzM10}, where 
 the crucial notions are the notion of monad and, more importantly, \emph{module over a monad}.
 It turns out that more work than a simple dualization is necessary, for two reasons which we indicate in the following.
 
 

%  give an extension of the characterisation as terminal coalgebra to an instance of \emph{heterogeneous} coinductive types,
%  namely the coinductive datatype of infinite triangular matrices as studied by
%  \textcite{DBLP:conf/types/MatthesP11}. 
 
 The codata type family $\Tri$ of infinite triangular matrices is parametrized by a fixed set for entries not on the diagonal, 
 and indexed by another, \emph{variable}, set for entries on 
 the diagonal. The respective types of its specifying destructors $\head$ and $\tail$ are given in \Cref{fig:tri_destructors}.
 A first difference to the inductive case of the lambda calculus springs to mind:
 while the lambda calculus is essentially a monad---hence an \emph{endo}functor---on sets, the codata type $\Tri$ cannot be considered as such;
 for a given set $A$, the type $\Tri A$, equiped with a suitable \fat{bisimulation} relation, 
 constitutes a \emph{setoid}---a set equiped with an equivalence relation---rather than a set.
 Instead of considering comonads---as the dual notion to monads---we thus have to consider a notion of comonad whose underlying functor is not necessarily endo.
 This is precisely the raison d'\^etre of \emph{relative} comonads (defined below)---the dual notion to the relative monads defined in 
  \parencite{DBLP:conf/fossacs/AltenkirchCU10}. 
 Fortunately, relative comonads have all the nice properties we could ask for; 
 in particular, the notion of module over a monad of \parencite{DBLP:journals/iandc/HirschowitzM10} extends to modules over \emph{relative} monads
 \parencite{ahrens_relmonads}, which can be dualized to obtain comodules over relative comonads.
 
 The other, more important difference between the inductive and the coinductive case stems from \emph{heterogeneity}:
 Two dual operations are responsible for the heterogeneity of the data type of lambda terms and the codata type of infinite triangular
 matrices: while we precompose with taking a \emph{co}product in the case of the abstraction of the lambda calculus---in order to account
 for context extension when passing under a binder---,
 we precompose with taking a \emph{product} when giving the target of the $\tail$ destructor of $\Tri$.
 However, as it turns out, the treatment of the $\tail$ destructor is far more complicated than a dualization of that of the abstraction 
 constructor of the lambda calculus---see \Cref{rem:shift} for a detailed explanation of the difference. 
 Describing our approach to a category-theoretic treatment of the $\tail$ destructor is the main purpose of the present work.
 
%  Our extension is in the spirit of the work by \textcite{DBLP:journals/iandc/HirschowitzM10}, which uses the notion of \emph{monad} and
%  \emph{module over a monad} to characterize substitution.
%  However, in order to characterize the cosubstition---\emph{redecoration}---of infinite triangular matrices, 
%  a simple dualisation of that work is not sufficient: since coinductive datatypes such as the aforementioned matrices 
%  live naturally in the category of 
%  \emph{setoids}, i.e.\ of sets equiped with an equivalence relation---the \fat{bisimulation} relation---rather than in the category of sets,
%  comonads are not good enough; we need a notion of comonad where the underlying functor is not necessarily endo.
%  
%  The notion of module over a monad was generalized to relative monads in \parencite{ahrens_relmonads}.
%  In the present work, we dualize this generalization to obtain comodules over relative comonads, and use this gadget to 
%  give a category-theoretic characterisation of redecoration for infinite triangular matrices.
%  
%  At first glance, infinite triangular matrices look dual to, say, the lambda calulus as a heterogeneous datatype \parencite{DBLP:journals/iandc/HirschowitzM10}
%  in every aspect: 
%  . We have already mentioned a first aspect that makes the coinductive case more complicated: the need for 
%  \emph{relative} comonads rather than traditional comonads.
%  
%  Our framework should generalize to heterogeneous coinductive datatypes specified by signatures in the spirit of \parencite{ahrens_relmonads},
%  but we lack interesting examples other than the considered triangular matrices.
 
 All our definitions, examples and lemmas have been implemented in the proof assistant \coq \parencite{coq84pl3}. 
 In this document, we hence omit the proofs and focus on definitions and statements of lemmas.
 
\subsection*{Organisation of the paper}
  In \Cref{sec:preliminaries} we introduce some concepts and notations used later on.
  In \Cref{sec:comonads} we define the category of comonads relative to a fixed functor.
  In \Cref{sec:comodules} we define comodules over relative comonads and give some constructions of comodules.
  In \Cref{sec:coalgebras_for_tri} we explain an important difference between the lambda calculus and infinite triangular matrices
    as heterogeneous data types. We then set up the necessary definitions to state our main theorem (\Cref{ex:final_sem_tri}), 
    characterizing the codata type $\Tri$ as terminal object in some category.
  In \Cref{sec:formal} we explain some details of the formalization of this work in the proof assistant \coq.
 
\section{Preliminaries}\label{sec:preliminaries}


\begin{definition}[Some categories]\label{def:set_setoid}
 We denote by $\Set$ the category of sets and total functions. 
 
 We denote by $\Setoid$ the category an object of which is a set equiped with an equivalence relation.
 A morphism between setoids is a set-theoretic function between the underlying sets that is compatible with the equivalence relations of the source and target setoids.
 Given a setoid $T$, we write $t\sim t'$ for related elements $t$ and $t'$ in $T$.
\end{definition}



\begin{definition}\label{def:eq}
 We define the functor $\eq : \Set\to\Setoid$ as the left adjoint to the forgetful functor 
  $U : \Setoid \to \Set$.
  Explicitly, the functor $\eq$ sends any set $X$ to the setoid $(X,=_X)$ given by the set $X$ itself, together
  with the equality relation $=_X$ on $X$.
\end{definition}


\begin{remark}[Notation for product]
  We denote the category-theoretic binary product of objects $A$ and $B$ of a category $\C$ by $A\times B$.
  We write $\pr_1(A,B) : \C(A\times B, A)$ and $\pr_2(A,B) :\C(A\times B, B)$ for the projections, and we omit the 
  argument $(A,B)$ when it is clear from the context.
  Given $f : \C(A, B)$ and $g : \C(A,C)$, we write $\langle f,g\rangle : \C(A,B\times C)$ for the induced map into the product such that
  $\comp{\langle f,g\rangle}{\pr_1} = f$ and $\comp{\langle f,g\rangle}{\pr_2} = g$.
\end{remark}

Both of the categories of \Cref{def:set_setoid} have binary products; they are \emph{cartesian monoidal}, i.e.\ the terminal 
object is neutral with respect to the multiplication. Functors preserving the monoidal structure \textbf{up to isomorphism}
are called \emph{strong monoidal}:

\begin{definition}[Strong monoidal functor]\label{def:monoidal_functor}
 A functor $F:\C\to\D$ between cartesian monoidal categories is \fat{strong monoidal} if, for any two objects $A$ and $B$ of $\C$,
  the morphism
 \[ \alpha^F_{A,B} := \bigl\langle F(\pr_1), F(\pr_2) \bigr\rangle : \D\bigl(F(A\times B), FA\times FB\bigr)\enspace  \] 
 is an isomorphisms.
 (Note that for \emph{cartesian} monoidal categories, the family $\alpha$ of morphisms automatically makes commute the 
  diagrams necessary to have a strong monoidal functor, since it is given by universal property. 
  We shall not need more general strong monoidal functors between arbitrary monoidal categories.)
\end{definition}

\begin{example}
  The functor $\eq: \Set \to \Setoid$ of \Cref{def:eq} is strong monoidal.
\end{example}


\section{Relative comonads and their morphisms}\label{sec:comonads}

\emph{Relative monads} were defined by \textcite{DBLP:conf/fossacs/AltenkirchCU10} as a notion of monad-like structure
whose underlying functor is not necessarily an endofunctor.
The dual notion is that of a relative \emph{co}monad:

\begin{definition}[Relative comonad]\label{def:rel_comonad}
  Let $F:\C\to\D$ be a functor. A \fat{relative comonad $T$ over $F$} is given by
  \begin{packitem}
   \item a map $T:\C_0 \to \D_0$ on the objects of the categories involved;
   \item an operation $\counit : \forall A : \C_0, \D(TA,FA)$;
   \item an operation $\cobind: \forall A,B:\C_0, \D(TA,FB) \to \D(TA,TB)$
  \end{packitem}
  such that 
  \begin{packitem}
   \item $\forall A,B:\C_0, \forall f:\D(TA,FB), \comp{\cobind(f)}{\counit_B} = f$;
   \item $\forall A : \C_0, \cobind(\counit_A) = \id_A$;
   \item $\forall A,B,C:\C_0, \forall f : \D(TA,FB),\forall g:\D(TB,FC), \\
        \comp{\cobind(f)}{\cobind(g)} = \cobind(\comp{\cobind(f)}{g})$.
  \end{packitem} 
\end{definition}
Just like relative monads, relative comonads are functorial:
\begin{definition}[Functoriality for relative comonads]\label{def:lift}
 Let $T$ be a  comonad relative to $F:\C\to\D$.
 For $f : \C(A,B)$ we define
  \[ \lift^T(f) := \cobind(\comp{\counit_A}{Ff}) .  \]
 The functor properties are easily checked.
\end{definition}
Relative comonads over the identity functor are exactly comonads.
A nontrivial instance of relative comonads is given by the coinductive datatype of infinite triangular matrices
studied by \textcite{DBLP:conf/types/MatthesP11}:


\begin{example}[\textcite{DBLP:conf/types/MatthesP11}]\label{ex:tri_comonad}
Let $E$ be an arbitrary fixed set.
   Infinite triangular matrices over the (variable) type $A$ are given by the destructors shown in \Cref{fig:tri_destructors},
    equiped with a coinductively defined equivalence relation, a \emph{bisimulation}.
    We write $t \sim t'$ for related matrices $t$ and $t'$.
%   \begin{lstlisting}
% Coinductive Tri (E : Set) : Set -> Set :=
%    head : forall A : Set, Tri (A) -> A.
%  | tail : forall A : Set, Tri (A) -> Tri (E x A).
%   \end{lstlisting}
% 

\begin{figure}[hbt]
  \begin{center}
    % \begin{minipage}{20em}
     \def\extraVskip{3pt}
     \def\proofSkipAmount{\vskip.8ex plus.8ex minus.4ex}
    \AxiomC{$t : \Tri(A)$}\doubleLine
     \UnaryInfC{$\head_A(t) : A$}
      \DisplayProof
                        \hspace{3ex}
                                       \AxiomC{$t : \Tri(A)$}\doubleLine
                                       \UnaryInfC{$\tail_A(t) : \Tri(E\times A)$}
                                       \DisplayProof%
     % \end{minipage} 
  \end{center}
\vspace{2ex}
  \begin{center}
                                            \def\extraVskip{3pt}
     \def\proofSkipAmount{\vskip.8ex plus.8ex minus.4ex}
    \AxiomC{$t \sim t'$}\doubleLine
     \UnaryInfC{$\head(t) = \head(t')$}
      \DisplayProof
                        \hspace{3ex}
                                       \AxiomC{$t \sim t'$}\doubleLine
                                       \UnaryInfC{$ \tail(t) \sim \tail(t')$}
                                       \DisplayProof   
  \end{center}
  \caption{Destructors and bisimulation for the coinductive family $\Tri$} \label{fig:tri_destructors}
\end{figure}
% \begin{figure}[tb]
%   \centering
% %   \fbox{
%                                             \def\extraVskip{3pt}
%      \def\proofSkipAmount{\vskip.8ex plus.8ex minus.4ex}
%     \AxiomC{$t \sim t'$}\doubleLine
%      \UnaryInfC{$\head(t) = \head(t')$}
%       \DisplayProof
%                         \hspace{3ex}
%                                        \AxiomC{$t \sim t'$}\doubleLine
%                                        \UnaryInfC{$ \tail(t) \sim \tail(t')$}
%                                        \DisplayProof
%      % \end{minipage} 
% %        }
%   \caption{Bisimulation relation for $\Tri$} \label{fig:tri_bisim}
% \end{figure}
%

\noindent
\textcite{DBLP:conf/types/MatthesP11} define a notion of \emph{weak constructive comonad}, and show that the datatype $\Tri$
  is an instance of such a comonad. 
  Such a weak constructive comonad, however, is \emph{precisely}
  a comonad relative to the functor $\eq : \Set\to\Setoid$.
  The $\counit$ is given by the operation $\head$.
  The $\cobind$ operation---called \emph{redecoration} or $\redec$ in \parencite{DBLP:conf/types/MatthesP11}--- of type
   \[ \redec_{A,B} : \Setoid(\Tri A,\eq B) \to \Setoid(\Tri A,\Tri B )\] is defined 
  corecursively, for $f:\Setoid(\Tri A,\eq B)$  and $t : \Tri A$ by
      \begin{align}\head\bigl(\redec~f~t\bigr) &:= f~t \quad\text{ and } \notag\\
                     \tail\bigl(\redec~f~t\bigr) &:= \redec\bigl(\extend~f\bigr)(\tail~t) \enspace . \label{eq:rest_redec}
      \end{align}
Here, the family of functions 
     \[\extend_{A,B} : \Setoid(\Tri A,\eq B) \to \Setoid\bigr(\Tri (E \times A),\eq(E\times B)\bigr)\]
  is suitably defined to account for the change of the type of argument of $\redec$ when redecorating $\tail~t : \Tri(E\times A)$
  rather than $t : \Tri A$, namely
  \[ \extend(f) := \langle \comp{\head_{E\times A}}{\pr_1(E,A)} , \comp{\cut_A}{f} \rangle \enspace . \]
  The auxiliary function $\cut_A : \Tri(E\times A) \to \Tri A$ is defined corecursively via
   \begin{align*}\head\bigl(\cut~t\bigr) &:= \pr_2(\head~t) \quad\text{ and } \\
                     \tail\bigl(\cut~t\bigr) &:= \cut(\tail~t) \enspace . 
      \end{align*}
 Note that what is called $\extend$ here is called \textit{lift} in \parencite{DBLP:conf/types/MatthesP11}.
\end{example}






\begin{definition}[Morphism of relative comonads]\label{def:comonad_morphism}
 Let $T$ and $S$ be comonads relative to a functor $F : \C \to \D$. A \fat{morphism of relative comonads} $\tau : T \to S$
  is given by a family of morphisms \[\tau_A : \D(TA,SA)\] such that for any $A : \C_0$,
     \[  \counit^T_A = \comp{\tau_A}{\counit^S_A} \]
   and for any $A,B : \C_0$ and $f : \D(SA,FB)$,
   \[  \comp{\cobind^T(\comp{\tau_A}{f})}{\tau_B} = \comp{\tau_A}{\cobind^S(f)} \enspace .  \]
\end{definition}

A morphism $\tau : T\to S$ of relative comonads over a functor $F:\C\to\D$ is  \emph{natural}
with respect to the functorial action of \Cref{def:lift}.
% 
% \begin{remark}[Category of relative comonads]
%  Fix a functor $F : \C\to\D$. 
Relative comonads over a fixed functor $F$ and their morphisms form a category $\RComonad(F)$.
% \end{remark}

In the definition of the $\extend$ operation in \Cref{ex:tri_comonad} the auxiliary operation $\cut$ plays a central r\^ole.
It should be remarked here that, using functoriality of $\Tri$, one can define another operation of the same type, namely the family of 
maps $\Tri(\pr_2(E,A)) : \Tri(E\times A) \to \Tri A$ for
all sets $A$. Had we used this operation rather than the corecursively defined $\cut$ of \Cref{ex:tri_comonad}, 
then the third comonad law would have failed; we refer to \parencite{DBLP:conf/types/MatthesS07} for details.



\section{Comodules over relative comonads}\label{sec:comodules}

In this section we dualize the notion of module over a relative monad as developed in \parencite{ahrens_relmonads}.
This dualization is straightforward.


\begin{definition}[Comodule over relative comonad]\label{def:comodule}
 Let $T$ be a relative comonad over the functor $F:\C\to\D$, and let $\E$ be a category.
 A \fat{comodule over T towards $\E$} consists of
   \begin{packitem}
   \item a map $:\C_0 \to \E_0$ on the objects of the categories involved;
   \item an operation $\mcobind: \forall A,B:\C_0, \D(TA,FB) \to \E(MA,MB)$
  \end{packitem}
  such that 
  \begin{packitem}
   \item $\forall A : \C_0, \mcobind(\counit_A) = \id_A$;
   \item $\forall A,B,C:\C_0, \forall f : \D(TA,FB),\forall g:\D(TB,FC), \\
        \comp{\mcobind(f)}{\mcobind(g)} = \mcobind(\comp{\cobind(f)}{g})$ .
  \end{packitem}

\end{definition}

Every relative comonad comes with a canonical comodule over it, the \emph{tautological comodule}:

\begin{definition}[Tautological comodule]\label{def:tautological_comodule}
  Given a comonad $T$ relative to $F:\C\to\D$, the assignment $A \mapsto TA$ yields a comodule over $T$ 
  with target category $\D$, the \textbf{tautological comodule} of $T$, also called $T$.
  The comodule operation is given by
    \[  \mcobind^T(f) := \cobind^T(f) \enspace . \]
\end{definition}


A \emph{morphism of comodules} is given by a family of morphisms that is compatible with 
the comodule operation:

\begin{definition}[Morphism of comodules]\label{def:morphism_of_comodules}
 Let $M, N : \C \to \E$ be comodules over the comonad $T$ relative to  $F:\C \to \D$.
 A \fat{morphism of comodules} from $M$ to $N$ is given by a family of morphisms 
   \[ \alpha_A:\E(MA,NA) \]
 such that for any $A,B:\C_0$ and $f : \D(TA,FB)$ one has
 \[   \comp{\mcobind^M(f)}{\alpha_B} = \comp{\alpha_A}{\mcobind^N(f)} \enspace . \]
\end{definition}


Composition and identity of comodule morphisms happens pointwise. We thus obtain a category of comodules
over a fixed comonad, towards a fixed target category:

\begin{remark}[Category of comodules]
 Fix a comonad $T$ relative to $F:\C\to\D$ and a category $\E$.
 Comodules over $T$ towards $\E$ and their morphisms  form a category $\RComod(T,\E)$.
\end{remark}

Similarly to relative comonads, comodules over these are functorial:

\begin{definition}[Functoriality for comodules]\label{def:comodule_lift}
 Let $M : \RComod(T,\E)$ be a comodule over $T$ towards some category $\E$. For $f : \C(A,B)$ we define
  \[ \mlift^S(f) := \mcobind(\comp{\counit_A}{Ff}) .  \]
\end{definition}


\begin{remark}
  The family of morphisms constituting a comodule morphism is actually natural with respect to the functoriality 
  defined in \Cref{def:comodule_lift}.
\end{remark}

Given a morphism of comonads, we can \enquote{transport} comodules over the source comonad to comodules over the target comonad:


\begin{definition}[Pushforward comodule]\label{def:pushforward_comodule} % checked
  Let $\tau : T\to S$ be a morphism of comonads relative to a functor $F : \C \to \D$, and let furthermore $M$ be a 
  comodule over $T$ towards a category $\E$. We define the \fat{pushforward comodule} $\tau_*M$ to be the comodule over $S$ given by
  \[  \tau_*M(A) := MA \]
  and, for $f : \D(SA,FB)$,
   \[ \mcobind^{\tau_*M}(f) := \mcobind^M(\comp{\tau_A}{f}) \enspace . \]
   
  \noindent
  Pushforward is functorial: if $M$ and $N$ are comodules over $T$ with codomain category $\E$, and $\alpha : M\to N$ is 
    a morphism of comodules, then we define \[\tau_*\alpha : \tau_*M \to \tau_*N\] 
    as the family of morphisms
     \[ (\tau_*\alpha)_A := \alpha_A \enspace . \]
  It is easy to check that this is a morphism of comodules (over $S$) between $\tau_*M$ and $\tau_*N$.
  Pushforward thus yields a functor $\tau_*:\RComod(T,\E) \to \RComod(S,\E)$.
\end{definition}


\begin{definition}[Morphism of comonads induces morphism of comodules] % checked
  Let $\tau : T\to S$ be a morphism of comonads relative to a functor $F : \C \to \D$.
  Then $\tau$ gives rise to a morphism of comodules over $S$ from the pushforward of the tautological comodule
  of $T$ along $\tau$ to the tautological comodule over $S$,
  \[ \induced{\tau} : \tau_*T \to S \enspace , \quad \induced{\tau}_A := \tau_A \enspace . \]
\end{definition}




\section{Coalgebras for infinite triangular matrices}\label{sec:coalgebras_for_tri}


Our goal is to define \fat{coalgebras} for the signature of infinite triangular matrices such that 
the codata type $\Tri$ defined in \Cref{ex:tri_comonad} constitutes the \fat{terminal such coalgebra}.
Since $\Tri$ forms a comonad relative to the functor $\eq:\Set\to\Setoid$, it seems reasonable to define
an arbitrary coalgebra as such a relative comonad as well, equiped with some extra structure which remains to be defined.
A morphism of coalgebras would then be defined as a morphism of relative comonads that is compatible with that extra structure.

The purpose of this extra structure is to characterize the destructor $\tail$, that is, its \emph{type} as well as
its \emph{behaviour with respect to the redecoration operation}.



\begin{remark}[$\tail$ is not a morphism of comonads]
At first glance, one might hope for $\tail$ to constitute the carrier of a morphism of relative comonads, i.e.\ one might 
guess that the axioms of comonad morphism capture the commutativity of $\tail$ with the operations $\head$ and $\tail$:
  Let $T$ be a comonad relative to a strong monoidal functor $F:\C\to\D$ between cartesian monoidal categories,
  and let $E:\C_0$ be a fixed object of $\C$.
 The map $A\mapsto T(E\times A)$ inherits the structure of a comonad relative to $F$ from $T$: the 
 counit is defined as
   \[ \counit^{T(E\times \_)}_A := \comp{\counit^T_{E\times A}}{F\pr_2(E,A)} \]
  and the cobind operation as
   \begin{align*} 
            \cobind^{T(E\times \_)}_{A,B} : \D\bigl(T(E\times A),FB\bigr) &\to \D\bigl(T(E\times A),T(E\times B)\bigr) \\
%              \cobind^{T(E\times \_)}_{A,B}
              f &\mapsto  \cobind^T(\extend'~f)
   \end{align*}
  with $\extend'$ defined as 
  \begin{align*} \extend' : \D\bigl(T(E\times A),FB\bigr) &\to \D\bigl(T(E\times A), F(E\times B)\bigr) \enspace , \\ 
                                            f & \mapsto \comp{\langle \comp{T(\pr_1)}{\counit^T_E}, f \rangle}{{\alpha^{F}_{E,B}}^{-1}} \enspace .
  \end{align*}
 One might thus try to characterize the destructor $\tail$ of the triangular matrices of \Cref{ex:tri_comonad}
 as a morphism of relative comonads, of type $\Tri \to \Tri(E\times \_)$.
 However, this fails for similar reasons as for heterogeneous inductive datatypes, see \parencite[Ex.\ 3.18]{ahrens_zsido}.
 %TODO: give explicit counterexample
\end{remark}


We now take a closer look at how the $\extend$ operation for $\Tri$ is defined. Its purpose is to 
modify the function $f$ used for redecoration, when redecorating $\tail~t : \Tri(E\times A)$ rather than $t : \Tri A$.
First we note an important difference between the $\extend$ operation and its analogue in the inductive case:

\begin{remark}[$\shift$ operation for the lambda calculus]\label{rem:shift}
  The heterogeneous data type of lambda terms comes with a term constructor $\abs_A : \ULC(A^*) \to \ULC A$, which 
  is the constructor responsible for the heterogeneity of that data type.
  We compare the $\extend$ operation to the one with analogue purpose in the case of the lambda calculus,
  the $\shift$ operation, of type
  \begin{equation}  \shift_{A,B} : \Set(A,\ULC B) \to \Set\bigl(A^*,\ULC(B^*)\bigr) \enspace .\label{eq:shift}\end{equation}
  Here $\ULC A$ denotes the set of lambda terms in context $A$, and $A^*:=1 + A$ denotes the context $A$ extended by a distinguished 
  variable---the variable bound by the abstraction constructor.
  
  The $\shift$ operation is used in the definition of the monadic substitution operation
  \[ \subst_{A,B} : \Set(A,\ULC B) \to \Set(\ULC A,\ULC B) \]
  of the lambda calculus, when recursively substituting under an abstraction:
   \[ \subst~f~(\abs~t):= \abs\bigl(\subst~(\shift~f)~t\bigr) \enspace . \]
  The reader is encouraged to compare this equation to \Cref{eq:rest_redec}.
  
  The $\shift$ operation can be defined in monadic terms, for any monad $T$ on the category of sets such that,
  when $T$ is instantiated by the monad $\ULC$, it becomes the specific $\shift$ operation of \Cref{eq:shift}.
  
  However, we were \emph{not} able to express the $\extend$ operation of $\Tri$ entirely in terms of categorical and comonadic operations. Indeed, if
  one dualizes the categorical definition of $\shift$, one obtains an operation that has the same type as $\extend$, 
  but for which, when used to define the redecoration, some comonadic property of redecoration fails.
  More specifically, it is the auxiliary $\cut$ operation as defined in \Cref{ex:tri_comonad} which does not seem to be 
  definable in categorical terms. There is an operation of the same type which is categorical---see \Cref{canonical_cut}---but
  which, when used for the definition of the $\extend$ operation, yields a \enquote{redecoration} operation for which the third comonad law fails.
  Details are given in an article by \textcite{DBLP:conf/types/MatthesS07}.
\end{remark}


The failure to describe the $\extend$ operation---more precisely, the auxiliary $\cut$---in categorical terms leads us 
to considering comonads that come with a specified $\cut$ operation for some fixed object $E$ in the source category:


% In what follows, we fix an object $E$ of the category $\C$.


% For an arbitrary comonad $T$ relative to a suitable functor $F$, an analogue operation $\cut^T_A : \D(T(E\times A),TA)$
% does not seem to be deducible from 



\begin{definition}[Relative comonad with cut]\label{def:rel_comonad_with_cut}
 Let $\C$ and $\D$ be categories with binary products and $F:\C\to\D$ a strong monoidal functor. Let $E:\C_0$ be a fixed object of $\C$.
 We define a \fat{comonad relative to $F$ with cut relative to $E$} to be a comonad $T$ relative to $F$ together with a $\cut$ operation 
    \[ \cut : \forall~A:\C_0, T(E\times A) \to TA \]
 satisfying the following axioms:
  \begin{itemize}
   \item $\forall~A:C_0, \comp{\cut_A}{\counit_A} = \comp{\counit_{E\times A}}{F(\pr_2(E,A))}$;
   \item $\forall~A~B:C_0,\forall~f:\D(TA,FB), \comp{\cut_A}{\cobind(f)} = \comp{\cobind(\extend~f)}{\cut_B}$,
  \end{itemize}

  \noindent
  where, for $f:\D(TA,FB)$, we define $\extend(f) : \D\bigl(T(E\times A),F(E\times B)\bigr)$ as
       \[ \extend(f) := \comp{\comp{\langle T(\pr_1) , \cut \rangle}{(\counit_E\times f)}}{{\alpha^{F}_{E,B}}^{-1}} \enspace . \]
  
\end{definition}

Morphisms of comonads with cut are morphisms of comonads that are compatible with the respective $\cut$ operations:

\begin{definition}[Morphism of comonads with cut]\label{def:morphism_comonad_cut}
 Let $T$ and $S$ be two comonads relative to a functor $F$ with cut relative to $E$ as in \Cref{def:rel_comonad_with_cut}.
 A \emph{morphism of comonads with cut} is a comonad morphism $\tau$ between the underlying comonads as in \Cref{def:comonad_morphism} that 
 commutes suitably with the respective $\cut$ operations, i.e.\ for any $A : \C_0$,
  \[ \comp{\tau_{E \times A}}{\cut^S_A}  = \comp{\cut^T_A}{\tau_A} \enspace . \]
%  TODO: where is tau_cut used?
\end{definition}


Comonads with cut relative to a fixed functor $F:\C\to\D$ and $E:\C_0$ form a category $\RComonadWC(F,E)$.
There is the obvious forgetful functor from $\RComonadWC(F,E)$ to $\RComonad(F)$.
Conversely, any comonad $T$ relative to a suitable functor can be equiped with a $\cut$ operation, using functoriality of $T$:



\begin{remark}[Canonical $\cut$ operation]\label{canonical_cut}
 Any comonad $T$ relative to a functor $F$ as in \Cref{def:rel_comonad_with_cut} can be equiped with a $\cut$ operation by setting
   \[ \cut^T_A := \lift^T\bigl(\pr_2(E,A)\bigr) \enspace . \]
 It follows from the axioms of comonad morphism that a comonad morphism $\tau : T\to S$ satisfies the equation of \Cref{def:morphism_comonad_cut} 
 for the thus defined operations $\cut^T$ and $\cut^S$, hence constitutes a morphism of comonads with cut from $(T,\cut^T)$ to $(S,\cut^S)$.
\end{remark}

The functor given by the canonical $\cut$ operation, followed by the forgetful functor, yields the identity. We can thus view
relative comonads with cut as a generalization of relative comonads.

Our prime example of relative comonad comes with a $\cut$ operation that is not the canonical one:

\begin{example}[$\cut$ for $\Tri$]\label{def:cut_for_tri}
  The relative comonad $\Tri$ from \Cref{ex:tri_comonad}, together with the $\cut$ operation defined there, 
  is a comonad with cut as in \Cref{def:rel_comonad_with_cut}.
\end{example}















\begin{comment}
\begin{definition}[IS ALREADY IN rel comonad with cut]\label{def:extend}
 Let $F : \C\to \D$ be a strong monoidal functor between (cartesian) monoidal categories, 
 i.e.\ we have a family of isomorphisms
  \[ \alpha_{A,B} : F(A\times B) \xrightarrow{\sim} FA\times FB\enspace . \] 
  Let $T$ be a relative comonad over $F$, and let $E : \C$ be an object of $\C$.
 We define
 \begin{align*} \extend_{A,B} : \D(TA,FB) &\to \D\bigl(T(E\times A),F(E\times B)\bigr)\\
                                   f   &\mapsto \comp{\bigl\langle \comp{\counit_{E\times A}}{\comp{\alpha_{E,A}}{\pr_1(FE,FA)}}, \comp{\lift(\pr_2(E,A))}{f} \bigr\rangle}{\alpha_{E,B}^{-1}} \enspace .
\end{align*}
% with 
% \begin{align*}
%                \cut_A: \C(T(E\times A), TA), \quad \cut_A := \cobind_{E \times A, A}(\comp{\counit_{E \times A}}{\pr_1}) \enspace .
% \end{align*}

Equivalently,
  \[ f   \mapsto \comp{\comp{\langle T(\pr_1) , T(\pr_2) \rangle}{(\counit_E\times f)}}{\alpha_{E,B}^{-1}} \enspace .
  \]

  
Equivalently,
  \[ f   \mapsto \comp{\langle \comp{T(\pr_1)}{f} , F(\pr_2) \rangle}{\alpha_{E,B}^{-1}} \enspace .
  \]
  
Equivalently,
  \[f  \mapsto \comp{\bigl\langle \comp{\counit_{E\times A}}{F(\pr_1(E,A))}, \comp{\lift(\pr_2(E,A))}{f} \bigr\rangle}{\alpha_{E,B}^{-1}} \enspace .
  \]
\end{definition}
\end{comment}



Given a comodule over a relative comonad with cut, we define a new comodule over the same comonad, given by precomposition with
\enquote{product with a fixed object $E$}:


\begin{definition}[Precomposition with product]\label{def:product_in_context}
 Suppose $F:\C\to\D$ is a strong monoidal functor, and $T$ is a comonad relative to $F$ with a $\cut$ operation 
 relative to $E:\C_0$ as in \Cref{def:rel_comonad_with_cut}.
 Given a comodule $M$ over $T$,  precomposition with \enquote{product with $E$}
 gives a comodule $M(E\times\_)$ over $T$,
  \[    M(E\times \_) : A \mapsto M(E\times A) \enspace . \]
 The comodule operation is deduced from that of $M$ by 
 \begin{align*} \mcobind^{M(E\times\_)}_{A,B} : \D(TA,FB)&\to \E\bigl(M(E\times A), M(E\times B)\bigr) \enspace ,\\
                                                      f &\mapsto \mcobind^M_{E\times A,E\times B}(\extend(f)) \enspace ,
  \end{align*}                                        
where the $\extend$ operation is the one defined in \Cref{def:rel_comonad_with_cut}.
 
 Furthermore, given two comodules $M$ and $N$ over $\T$ with target category $\E$, and a comodule morphism $\alpha : M \to N$, then 
 we can define a comodule morphism \[\alpha(E\times \_) : M(E\times \_) \to N(E\times \_) \] by setting
          \[ \alpha(E \times \_)_A := \alpha_{E\times A} \enspace . \]
  
  \noindent
  We thus obtain an endofunctor on the category of comodules over $T$ towards $\E$,
   \[ M \mapsto  M (E\times \_) : \RComod(T,\E) \to \RComod(T,\E) \enspace . \]
\end{definition}


It easily follows that the operation $\cut : T\to T(E\times \_)$  of any comonad $T$ with cut is natural with respect to 
the functoriality of comodules defined in \Cref{def:comodule_lift}, where the target comodule is the tautological comodule of $T$ and
the source comodule is given by precomposition with product of the tautological comodule.




\begin{example}[$\tail$ is a comodule morphism]\label{ex:tail_comodule}
 Consider the comonad $\Tri$, equiped with the $\cut$ operation of \Cref{def:cut_for_tri}.
 The destructor \constfont{\tail} of \Cref{ex:tri_comonad} is a morphism of comodules over the comonad $\Tri$ 
  from the tautological comodule  $\Tri$ to $\Tri(E\times \_)$.
\end{example}







\begin{remark}[Pushforward commutes with product in context]\label{rem:prod_pullback_commute}
 Note that the constructions of \Cref{def:product_in_context} and \Cref{def:pushforward_comodule} commute in the sense that
 we have an isomorphism of comodules \[\tau_*(M(E\times \_)) \cong (\tau_*M)(E \times \_) \enspace . \]
\end{remark}


We now have all the ingredients to state (and prove) our main theorem:

\begin{theorem}[Coinitial semantics for triangular matrices with redecoration]\label{ex:final_sem_tri} % checked
   Let $E:\Set_0$ be a set.
   Let $\mathcal{T} = \mathcal{T}_E$ be the category where an object consists of
   \begin{itemize}
    \item a comonad $T$ over the functor $\eq:\Set\to\Setoid$ with $\cut$ relative to $E$ and
    \item a morphism $\tail$ of comodules over $T$ of type $T \to T(E\times \_)$
   \end{itemize}
   such that for any set $A$,
    \[ \comp{\cut_A}{\tail_A} = \comp{\tail_{E\times A}}{\cut_{E\times A}} \enspace . \]

%     TODO: should read   cut o rest = cut(E \times \_ ) o rest(E\times \_)
   
   
   A morphism between two such objects $(T,\tail^T)$ and $(S,\tail^S)$
%    
% %    pair $(T,t)$ of a relative comonad $T$ over the functor
% %    $\eq: \Set \to \Setoid$ together with a morphism of comodules $t : T \to T(E \times \_)$.
%    A morphism $\tau : (T,t) \to (S,s)$ 
%    
%    
   is given by a morphism of relative comonads with cut $\tau : T \to S$ such that
   the following diagram of comodule morphisms in the category $\RComod(S,\E)$ commutes,
   
%    \[     \comp{\tau_*t}{\tau(E\times \_)} = \comp{\tau}{s} \enspace , \]
%    i.e.\, diagrammatically,   
   \[ \begin{xy}
       \xymatrix{   \tau_*T  \ar[r]^{\tau_*(\tail^T)} \ar[d]_{\induced{\tau}}  &  **[r] \tau_*T (E\times \_ )\ar[d]^{\induced{\tau}(E\times \_)} \\
                    S  \ar[r]_{\tail^S}  &  **[r] S (E\times \_ ) \enspace .
        }
      \end{xy}
   \]

   \noindent
   Here in the upper right corner we silently insert an isomorphism as in \Cref{rem:prod_pullback_commute}.
   
   Then the pair $(\Tri, \tail)$ consisting of the relative comonad with cut $\Tri$ of \Cref{def:cut_for_tri} together with 
    the morphism of comodules $\tail$ of \Cref{ex:tail_comodule},
   constitutes the terminal---\enquote{coinitial}---object in this category.
   
\end{theorem}



\begin{comment}

\section{Some more comodules}

In the following we present some more constructions of comodules which can be used to specify the type of 
destructors.



% \begin{definition}[Constant comodule]
%   Given $e:\E_0$, the function $\C_0\to\E_0$ sending any object of $\C$ to $e$ is equiped with
%   the structure of a comodule over $T$.
% \end{definition}


% \begin{definition}[Product comodule]
%  Given two comodules $M,N$ over $T$, then the pointwise product is equiped with the structure of 
%  a comodule over $T$.
% \end{definition}


Precomposition with \enquote{product with $E$} as in \Cref{def:product_in_context} is a special case of the following
general construction:

\begin{definition}[Precomposition with a functor]
  Let $T$ be a comonad relative to $F:\C\to\D$. 
  Let $K : \C\to\C$ be an endofunctor, and $K' : \D\to\D$ be another one, and let 
  $\alpha : TK \to K'T$ and $\beta : K'F \to FK$ be natural transformations.
  We then define
  \begin{align*} \extend_K : \D(TA,FB) &\to \D(TKA,FKB)  \enspace , \\ 
                                  f  &\mapsto \comp{\alpha_{A}}{\comp{K'f}{\beta_B}}  \enspace .
  \end{align*}
%
  Any comodule $M$ over $T$ with target category $\E$ then yields a comodule $MK$ with comodule operation
  \[ \mcobind^{MK}(f) := \mcobind^M(\extend_K(f)) \enspace . \]
\end{definition}

\begin{definition}[Postcomposition with a functor]\label{def:postcomposition_functor}
  Let $T$ be a comonad relative to $F:\C\to\D$, and let $M$ be a comodule over $T$ with codomain category $\E$.
 Let $K : \E \to \X$ be a functor. Then we define the comodule $KM$ to be the comodule over $T$ with codomain category $\X$
  sending any object $A:\C_0$ to $K(MA)$.
  The comodule operation is defined by
  \[ \mcobind^{KM}(f) := K (\mcobind(f)) \enspace . 
  \]
\end{definition}

\begin{example}[Infinite branching trees]
  Let $B$ a set. We consider $B$ as a setoid via the functor $\eq:\Set\to\Setoid$, which we silently apply.
  The hom functor $Y \mapsto \Setoid(B,Y)$ is an endofunctor on the category $\Setoid$: for any two setoids $X,Y$, 
   the set of setoid morphisms $\Setoid(X,Y)$ is itself equiped with an equivalence relation.
   Branching is given by postcomposition (\Cref{def:postcomposition_functor}) with such a hom functor.
\end{example}
\end{comment}




\section{Formalization in \coq}\label{sec:formal}

All our definitions and theorems are mechanized in the proof assistant \coq \parencite{coq84pl3}.
The formalization of infinite triangular matrices is taken from the work by \textcite{DBLP:conf/types/MatthesP11},
and only slightly adapted to compile with the version of \coq we use.


In the following we explain some of our design choices for this mechanization
and point out differences between the pen-and-paper definitions and the mechanized ones.



\subsection{Implementation choices}



\subsubsection{Setoids for hom-sets}
We formalize categories to be given by a type of objects and a dependent type---indexed by pairs of objects---of morphisms,
equiped with suitable composition and identity operations satisfying appropriate axioms.
More precisely, the family of morphisms is given by a family of \emph{setoids}, where the setoidal equivalence relation on each
type of morphisms denotes the equality relation on these morphisms. This approach was first used by
\textcite{aczel_galois} in the proof assistant \texttt{LEGO}, and also by \textcite{concat}  in their library
of category theory in \coq.
Alternatively, we could have chosen to consider morphisms modulo propositional equality.

Indeed, the morphisms we consider---morphisms of comonads and comodules---are given by structures
bundling a lot of data and properties; in order to consider two such morphisms as equal, we usually only compare one field of the 
corresponding records. Furthermore, this field usually consists of a (dependent) function.
It would be rather cumbersome to reduce equality of two such records to extensional equality of one of their fields, 
necessitating the use of the axioms of propositional and functional extensionality.
Using setoids for morphisms instead seems to come with less overhead and conceptually cleaner.


% This choice is encouraged also by our use of a version of the \coq theorem prover which features \fat{universe polymorphism},
% allowing to use a definition of setoids where the carrier is a \emph{field} rather than a \emph{parameter}.
% % TODO: give code
% In comparison, in the work by \textcite{concat}, the definitions of setoids and categories had to be repeated in order to avoid
% universe inconsistencies.
% As a side note, the universe polymorphism in the \coq version we use also allows to define the category of categories, even though
% we do not make any use of that.

\subsubsection{Records vs.\ classes}
Two approaches to the formalization of mathematical structures have been used extensively in \coq: on the one hand, packaging structures
in \emph{record types}  in combination with use of \emph{canonical structures}, is used with success, e.g., in 
the formalisation of algebraic structure in the context of the proof of the Feit-Thompson theorem \parencite{DBLP:conf/tphol/GarillotGMR09}.
On the other hand, \textcite{DBLP:journals/mscs/SpittersW11} suggest the use of \emph{type classes}, in particular when multiple inheritance
is an issue.

In the present formalization, we decide to use records rather than classes, since the strongest argument for type classes---multiple inheritance---does 
not occur.
We make use of canonical structures in order for \coq to deduce instance of categories when we mention objects of a category; 
in particular, this is used to allow for overloading of the notation for morphisms of a category.
We can thus conveniently use the same arrow symbol to denote the type of morphisms between two comonads, between two comodules and so on.


\subsection{Formal vs.\ informal definitions}

The only noteworthy discrepancy between formal and informal definition arises in the definition of the codata type of infinite 
triangular matrices:
in \coq, coinductive types are specified through \fat{constructors} rather than \fat{destructors}.
However, one can easily define inverse functions and work in terms of destructors afterwards.
Work on the the declaration of coinductive types via destructors is done in \agda, cf.\ \parencite{DBLP:conf/popl/AbelPTS13}.
Other than that, the informal definitions correspond closely to the formalized ones---which might also stem from the fact that 
the formalization was used as a research tool, playing an essential r\^ole in the development of this work.

\section{Conclusions and future work}



We have given a category-theoretic characterisation, via a universal property, of infinite triangular matrices
equiped with a comonadic redecoration operation.

While a significant part of our work---surmounting the non-categoricity of the $\cut$ operation---seems to be specific to this particular codata type,
we believe that our work proves the suitability of the notion of relative (co)monads and (co)modules thereover for 
a categorical treatment of coinductive data types.


We plan to pursue two obvious lines of work:
Firstly, we will work on a suitable notion of \emph{signature} for the specification of coinductive data types.
Secondly, we hope to integrate \emph{equations} into the notion of signature, equations which 
will allow, e.g, considering branching trees modulo permutation of subtrees.
 

\subsection*{Acknowledgments}
 We thank Ralph Matthes and Paige North for awesome discussions about stuff.

\printbibliography

% \newpage
\appendix


\section{Correspondance of informal and formal definitions}\label{sec:table_formal_informal}

{

\Crefname{definition}{Def.}{Defs.}
\Crefname{theorem}{Thm.}{Thms.}
\Crefname{example}{Ex.}{Exs.}

\begin{center}
{\renewcommand{\arraystretch}{1.2}
\begin{tabular}{lll}
Informal & Reference & Formal \\ \hline
Category &  & \lstinline!Category!\\
Functor &  & \lstinline!Functor!\\
Relative comonad & \Cref{def:rel_comonad} & \lstinline!RelativeComonad!\\
Triangular matrices as comonad & \Cref{ex:tri_comonad} & \lstinline!Tri!\\
Comodule over comonad & \Cref{def:comodule} & \lstinline!Comodule!\\
% Morphism of comodules & \Cref{def:morphism_of_comodules}& \lstinline!Comodule.Morphism!\\
Tautological comodule & \Cref{def:tautological_comodule} &\lstinline!tcomod!\\

Pushforward comodule & \Cref{def:pushforward_comodule} & \lstinline!pushforward!\\
Induced comodule morphism &\Cref{def:induced} & \lstinline!induced_morphism!\\
Relative comonad with cut &\Cref{def:rel_comonad_with_cut} & \lstinline!RelativeComonadWithCut!\\
Precomposition with product & \Cref{def:product_in_context} &\lstinline!precomposition_with_product!\\
$\tail$ is comodule morphism &\Cref{ex:tail_comodule} & \lstinline!Rest!\\
Coalgebras of triangular matrices & \Cref{def:cat_tri} & \lstinline!TriMat!\\
Triangular matrices are terminal & \Cref{ex:final_sem_tri} & \lstinline!Coinitiality!\\
\end{tabular}
}
\end{center}


}

\end{document}


%TODO: other arities, e.g.
%    - T(A) ---->  (E -> T(A))
%    - http://www2.tcs.ifi.lmu.de/~rodrigue/docs/types.html


















