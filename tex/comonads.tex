

\pdfoutput=1

\documentclass{amsart}

\usepackage[protrusion=true,expansion=true]{microtype}

\usepackage{amsmath}

\usepackage[style=authoryear,
 maxnames=2,
 maxbibnames=99 ,                
 uniquename=init ,
]{biblatex}
% \bibliography{literature.bib}

\usepackage{ownstyle}


\author{nobody}

\title{Comodules over relative comonads}

\newcommand{\fat}[1]{\textbf{#1}}

\begin{document}


\begin{abstract}
  bla bla
\end{abstract}

\maketitle

% \tableofcontents


\begin{definition}[Relative comonad]
  Let $F:\C\to\D$ be a functor. A \fat{relative comonad $T$ over $F$} is given by
  \begin{packitem}
   \item a map $T:\C_0 \to \D_0$ on the objects of the categories involved;
   \item an operation $\counit : \forall c : \C_0, \D(Tc,Fc)$;
   \item an operation $\cobind: \forall c,d:\C_0, \D(Tc,Fd) \to \D(Tc,Td)$
  \end{packitem}
  such that 
  \begin{packitem}
   \item $\forall c,d:\C_0, \forall f:\D(Tc,Fd), \comp{\cobind(f)}{\counit_d} = f$;
   \item $\forall c : \C_0, \cobind(\counit_c) = \id_c$;
   \item $\forall a,b,c:\C_0, \forall f : \D(Ta,Fb),\forall g:\D(Tb,Fc), \\
        \comp{\cobind(f)}{\cobind(g)} = \cobind(\comp{\cobind(f)}{g})$.
  \end{packitem} 
\end{definition}

\begin{example}
  The triangular matrices of Matthes \& Picard form a relative comonad over the functor from sets to setoids
  associating to any set the setoid together with the equality equivalence relation.
\end{example}


\begin{definition}[Functoriality for relative comonads]
 For $f : \C(c,d)$ we define
  \[ \lift^T(f) := \cobind(\comp{\counit_c}{Ff}) .  \]
  %TODO: prove functor laws
\end{definition}

\begin{definition}[Module over relative comonad]
 Let $T$ be a relative comonad over the functor $F:\C\to\D$, and let $\E$ be a category.
 A \fat{module over T} consists of
   \begin{packitem}
   \item a map $:\C_0 \to \E_0$ on the objects of the categories involved;
   \item an operation $\mcobind: \forall c,d:\C_0, \D(Tc,Fd) \to \E(Mc,Md)$
  \end{packitem}
  such that 
  \begin{packitem}
   \item $\forall c : \C_0, \mcobind(\counit_c) = \id_c$;
   \item $\forall a,b,c:\C_0, \forall f : \D(Ta,Fb),\forall g:\D(Tb,Fc), \\
        \comp{\mcobind(f)}{\mcobind(g)} = \mcobind(\comp{\cobind(f)}{g})$ .
  \end{packitem}

\end{definition}


\begin{definition}[Functoriality for modules over relative comonads]
 Let $M$ be a comodule over $T$. For $f : \C(c,d)$ we define
  \[ \mlift^S(f) := \mcobind(\comp{\counit_c}{Ff}) .  \]
  %TODO: functoriality for rel comonads 
\end{definition}


\begin{definition}[Constant comodule]
  Given $e:\E_0$, the function $\C_0\to\E_0$ sending any object of $\C$ to $e$ is equiped with
  the structure of a comodule over $T$.
\end{definition}


\begin{definition}[Product comodule]
 Given two comodules $M,N$ over $T$, then the pointwise product is equiped with the structure of 
 a comodule over $T$.
\end{definition}

\begin{definition}
 Suppose $\C$ has binary products. Let $E : \C$
 We define
 \begin{align*} \lift_{A,B} : \C(TA,B) &\to \C(T(E\times A),E\times B)\\
                                   f   &\mapsto \langle \comp{\counit_{E\times A}}{\pr_1}, \comp{\cut_{E\times A}}{f} \rangle
\end{align*}
with 
\begin{align*}
               \cut_A: \C(T(E\times A), TA), \quad \cut_A := \cobind_{E \times A, A}(\comp{\counit_{E \times A}}{\pr_1})
\end{align*}



\end{definition}



\begin{definition}[Product in context]
 Suppose $\C$ has binary products.
 Given a comodule $M$ over $T$ and an object $E:\C_0$, then precomposition with \enquote{product with $E$}
 gives a comodule $M(E\times\_)$ over $T$.
 
 The comodule operation is deduced from that of $M$ by 
 \begin{align*} \mcobind^{M(E\times\_)}_{A,B} : \C(TA,B)&\to \C(M(E\times A), M(E\times B)) \\
                                                      f &\mapsto \mcobind^M_{E\times A,E\times B}(\lift(f))
  \end{align*}                                        
 
 
%  Needs a shift operation $\D(Tc,Fd) \to \D(T(c\times c_0),F(d\times c_0))$
\end{definition}

\begin{definition}[Identity comodule]

  WRONG, it isn't 
  
  i don't delete this yet

 %TODO: identity comodule (i.e. identity functor forms a comodule over any relative comonad
 
 %
 
\end{definition}


\begin{definition}[Morphism of comodules]
 %TODO: morphism of comodules
\end{definition}


\begin{example}
 The constructor of triangular matrices is a comodule morphism of appropriate type.
\end{example}


% \printbibliography

\end{document}
