\section{Conclusion}

In this paper we have presented various verifications of the
redecoration algorithm for infinite triangles. We have first dualized
directly the representation for finite triangles by a nested inductive
datatype to obtain a nested coinductive datatype. In both cases, the
triangles are visualized by their columns. We have implemented the
corresponding redecoration algorithm \redec{} (already available \cite{grossestcspaper} in
higher-order parametric polymorphism) and
shown that we (only) obtained a constructive weak comonad (because of
the compatibility hypothesis required). In this part, the redecoration
algorithm, although deduced directly from the finite case, is quite
tricky to manipulate because of the cutting and lifting it requires.

We then noticed that we could also consider
the triangles by their rows, representing this time the triangles by
purely coinductive datatypes, \TriS{}, where we only took advantage of the
existing type of streams (\Str). This new visualization allowed us to
define -- keeping the same algorithmic idea as before -- a function of
redecoration \redecS{} already simpler than \redec{} and equivalent to
it, modulo compatibility. But taking advantage of this
representation by streams, we can simplify again the redecoration algorithm,
using only standard functions on streams. This new function
\redecSS{} is fully equivalent to \redecS{}. Generalizing again, we
get nearly for free the cobind of the comonad \Str{}, \redecSG{}. This
finally allows us to prove the three comonad laws for \redecSS{}. 

In short, we have shown that the redecoration function, which is a
quite subtle operation if we translate it directly from the finite
triangles, reduces to something very basic in the completely infinite
(i.\,e., in both directions) view of the infinite
triangles. 
In this case, it is much easier to work with only infinite elements
than with partially finite ones in the sense of consisting of
infinitely many finitely presented columns. In fact, the stream
representation is even easier to manipulate than the representation of
finite triangles, and the comonad laws even hold with less
restrictions due to constructivity. 

Notice that the row-based view would 
not have given new insights for finite triangles. Indeed, as they are
symmetric, we would have obtained exactly the same representation as
for the column-based approach, only perceived with interchanged roles
of rows and columns.

As a final remark on the Coq side, the improved support for setoid
rewriting and the class mechanism \cite{DBLP:conf/tphol/SozeauO08} has
shown to be of great help for the formalization and verification
decribed in this paper.



%%% Local Variables:
%%% mode: latex
%%% TeX-master: "coredec"
%%% End:
