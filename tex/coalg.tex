

\pdfoutput=1

\documentclass[envcountsame]{llncs}

\usepackage[utf8]{inputenc}

\usepackage[protrusion=true,expansion=true]{microtype}


\usepackage[style=numeric,
%  backref=true,
 isbn=false,
 maxnames=3,
 maxbibnames=99 ,                
 uniquename=init ,
]{biblatex}
\bibliography{literature.bib}

\usepackage{ownstylesansthm}

\usepackage{comment}
\excludecomment{Long}\includecomment{Short}
% \includecomment{Long}\excludecomment{Short}

\pagestyle{plain}

\begin{document}

\title{Non-wellfounded trees\\in intensional Martin-L\"of type theory}

\author{Benedikt Ahrens and R\'egis Spadotti}

\institute{
Institut de Recherche en Informatique de Toulouse\\
Universit\'e Paul Sabatier, 
Toulouse}

\newcommand{\fat}[1]{\textbf{#1}}
\newcommand{\M}{\constfont{M}}
\newcommand{\W}{\constfont{W}}
\renewcommand{\root}{\constfont{root}}
\newcommand{\br}{\constfont{br}}
\newcommand{\transport}{\constfont{transport}}


\maketitle

% \tableofcontents

\begin{abstract}


 In this work, we give a category-theoretic characterisation of \textsf{M}-types in intensional Martin-L\"of type theory.
 Our results are mechanized in the proof assistant \coq.
   
  
  \end{abstract}


\section{Introduction}



An inductive type in type theory is characterized as initial algebra of some functor.
Dually, a \emph{co}inductive type is characterized as final coalgebra.

More precisely, one usually considers trees---well-founded ones and non well-founded ones---of a fixed shape.
The shape of such trees is given by two pieces of data: firstly, a type $A$ of \emph{nodes}, and secondly,
a dependent type $B : A \to \Set$ associating to any node $a : A$ a type which specifies the subtrees of $a$.
The type of \emph{finite}---well-founded---trees with shape specified by $A$ and $B$ is called $\W_{A,B}$,
and the type of \emph{infinite} such trees is called $\M_{A,B}$.
The defining functor $P_{A,B} : \Set \to \Set$ is given by 
\[ P_{A,B}(X) := \sum_{a:A}\left(B(a) \to X\right) \enspace . \]



\section{Preliminaries}\label{sec:preliminaries}

In this section we present some particular categories and functors used later on, and fix some notation.


\begin{definition}[Some categories]\label{def:set_setoid}
 We denote by $\Set$ the category of types (of a fixed universe) and total functions between them in Martin-L\"of type theory. 
 A morphism $f$ in this category is denoted by $f : A \to B$.
 
 We denote by $\Setoid$ the category an object of which is a \emph{setoid}, i.e.\ a type equipped with an equivalence relation.
 A morphism between setoids is a type-theoretic function between the underlying types that is compatible in the obvious sense with the equivalence relations of the source and target setoids.
 If $A$ is a setoid, we also use $A$ to refer to its underlying type, and thus write $a:A$ for an element $a$ of the type underlying the setoid $A$. 
 We write $a\sim a'$ for related elements $a$ and $a'$ in $A$.
 We consider two parallel morphisms of setoids $f,g:A\to B$ equal if for any $a:A$ we have $fa \sim ga$.
 
 We also write $f:A\to B$ for a morphism $f$ between objects $A$ and $B$ in some category, in particular in the category of types.
 \end{definition}



\begin{definition}\label{def:eq}
 The functor $\eq : \Set\to\Setoid$ is defined as the left adjoint to the forgetful functor $U : \Setoid \to \Set$.
  Explicitly, the functor $\eq$ sends any type $X$ to the setoid $(X,=_X)$ given by the type $X$ itself, equipped
  with the propositional equality relation $=_X$ specified via Martin-L\"of's identity type on $X$.
\end{definition}


\begin{remark}[Notation for product]
  We denote the category-theoretic binary product of objects $A$ and $B$ of a category $\C$ by $A\times B$.
  We write $\pr_1(A,B) : \C(A\times B, A)$ and $\pr_2(A,B) :\C(A\times B, B)$ for the projections, occasionally omitting the 
  argument $(A,B)$.
  Given $f : \C(A, B)$ and $g : \C(A,C)$, we write $\langle f,g\rangle : \C(A,B\times C)$ for the induced map into the product such that
  $\comp{\langle f,g\rangle}{\pr_1} = f$ and $\comp{\langle f,g\rangle}{\pr_2} = g$.
\end{remark}

Both of the categories of \Cref{def:set_setoid} have binary products; they are \emph{cartesian monoidal}, i.e.\ the terminal 
object is neutral with respect to the product. Functors preserving the monoidal structure up to isomorphism
are called \emph{strong monoidal}:

\begin{definition}\label{def:monoidal_functor}
 A functor $F:\C\to\D$ between cartesian monoidal categories is \fat{strong monoidal} if, for any two objects $A$ and $B$ of $\C$,
  the morphism
 \[ \phi^F_{A,B} := \bigl\langle F(\pr_1), F(\pr_2) \bigr\rangle : \D\bigl(F(A\times B), FA\times FB\bigr)\enspace  \] 
 is an isomorphism.
 (Note that for \emph{cartesian} monoidal categories, the family $\phi$ of morphisms automatically 
  is compatible with the unitators and associators of the source and target categories, 
  since it is given by a universal property.)
\end{definition}

\begin{example}
  The functor $\eq: \Set \to \Setoid$ of \Cref{def:eq} is strong monoidal.
\end{example}


\section{Rules for $\M$-types in intensional Martin-L\"of type theory}\label{sec:rules}

We consider types of non-wellfounded trees of a given shape in Intensional Martin-L\"of type theory (IMLTT) \parencite{martin_lof}, 
a type-theoretic foundational system.
For $a,b : A$, we denote by $a = b$ the Martin-L\"of identity type between $a$ and $b$.


Suppose $A$ is a type in (intensional) Martin-L\"of type theory, and $B : A \to \Set$ is a dependent type.
The type $\M_{A,B}$ denotes trees---possibly infinite---with nodes 
from $A$ such that a node $a:A$ has \enquote{$B(a)$-many} subtrees.

The correct notion of \enquote{sameness} for non-wellfounded trees is \emph{bisimilarity} 
\parencite{DBLP:conf/types/Coquand93, DBLP:journals/corr/abs-cs-0603119},
an equivalence relation which identifies \enquote{observationally equal} trees.
A coinductive type with bisimilarity forms a setoid as in \Cref{def:set_setoid}.
We thus denote bisimilar elements using an infix $\sim$, as in $t \sim t'$. 

Here are the rules:



\begin{description}

 \item[Formation]\hfill \\
 
 \begin{center}
 \def\extraVskip{3pt}
     \def\proofSkipAmount{\vskip.8ex plus.8ex minus.4ex}

         
   \AxiomC{$A : \Set$} \AxiomC{$B: A \to \Set$}
    \BinaryInfC{$\M_{A,B} : \Set$}
     \DisplayProof
 \end{center} 
 
  \begin{center}
 \def\extraVskip{3pt}
     \def\proofSkipAmount{\vskip.8ex plus.8ex minus.4ex}

         
    \AxiomC{$s, t : \M_{A,B}$}
    \UnaryInfC{$s \sim t : \Set$}
     \DisplayProof
 \end{center} 
 
 
 \item[Destruction]\hfill \\
 

%   \hspace{3ex}

\begin{center}
    \AxiomC{$t : \M_{A,B}$} %\doubleLine
     \UnaryInfC{$\root(t) : A$}
      \DisplayProof
                        \hspace{3ex}
                                       \AxiomC{$t : \M_{A,B}$}%\doubleLine
                                       \UnaryInfC{$\br(t) : B(\root(t)) \to \M_{A,B}$}
                                       \DisplayProof%
\end{center}

\begin{center}
    \AxiomC{$s, t : \M_{A,B}$} \AxiomC{$p: s \sim t$} %\doubleLine
     \BinaryInfC{$\sim \root(p):\root(s) = \root(t)$}
      \DisplayProof
                        \hspace{3ex}                                      
                                      \AxiomC{$s,t : \M_{A,B}$} \AxiomC{$p: s \sim t$}%\doubleLine
                                       \BinaryInfC{$\sim\br(p): \forall~x, \transport(\br(s))(x) \sim \br(t)(x)$}
                                       \DisplayProof%
\end{center}


  \item[Creation]\hfill \\                                     
                       
%           \vspace{1em}             
            
\begin{center}
               \AxiomC{$T : \Set$} \AxiomC{$hd : T \to A$} \AxiomC{$tl : \forall~t:T, B(hd(t))\to T$} %\doubleLine
               \TrinaryInfC{$\corec_A~hd~tl : T \to \M_{A,B}$}
               \DisplayProof%
\end{center}

\begin{center}
               \AxiomC{$R : \M_{A,B} \to \M_{A,B} \to \Set$} \noLine
               \UnaryInfC{$\forall~s,t : \M_{A,B}, R~s~t \to \root(s) = \root(t)$} \noLine
               \UnaryInfC{$\forall~s,t : \M_{A,B}, \forall~p : \root(s) = \root(t), 
                               \forall~x : B(\root(s))(x) \sim \transport_p (B(\root(t)))(x)$}  %\doubleLine
               \UnaryInfC{$\forall~s,t : \stream A, R~s~t \to \bisim~s~t$}
               \DisplayProof%
\end{center}

                      
%           \vspace{1em}
  \item[Computation]\hfill \\

\begin{center}          
               \AxiomC{$hd : T \to A$} \AxiomC{$tl : \forall~t:T, B(hd(t))\to T$} \AxiomC{$t:T$}%\doubleLine
               \TrinaryInfC{$\root(\corec~hd~tl~t) = hd(t)$}
               \DisplayProof
               
               \vspace{1em}
               
               \AxiomC{$hd : T \to A$} \AxiomC{$tl : \forall~t:T, B(hd(t))\to T$} \AxiomC{$t:T$}%\doubleLine
               \TrinaryInfC{$\forall~x : B(hd(t)), \br(\corec~hd~tl~t)(x) \sim \corec~hd~tl~(\br(t)(x))$}
               \DisplayProof
\end{center}

 \end{description}              
               


\section{The category of coalgebras}



\begin{definition}\label{def:poly_functor}
 Suppose $A : \Set$ is a type and $B: A \to \Set$ is a dependent type.
 We define a functor $F_{A,B} : \Setoid \to \Setoid$ as follows:
 the setoid $(X,R)$ is mapped to

 the setoid with carrier $\sum_{a:A}(B(a) \to X)$ and setoid relation
 %TODO
 
 A setoid morphism $f : (X,Rx) \to (Y,Ry)$ is mapped to 
 \[F_{A,B}(f) (t):= \langle \pi_1(t), \comp{\pi_2(t)}{t} \rangle \enspace . \]
%  TODO: show this is setoid morph
 
 
\end{definition}





\section{Equivalence between the rules and existence of a terminal object}

\begin{lemma}
  Given $A:\Set$ and $B:A\to\Set$ in IMLTT, the following are equivalent:
%  TODO: what principles are needed?
 \begin{itemize}
  \item The axioms given in \Cref{sec:rules}.
  \item The existence of a terminal coalgebra for the functor $F_{A,B}$ as defined in \Cref{def:poly_functor}.
 \end{itemize}

\end{lemma}

 
\printbibliography


\appendix



\end{document}



















