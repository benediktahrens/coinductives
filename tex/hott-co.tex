

\pdfoutput=1

\documentclass[envcountsame]{llncs}

\usepackage[utf8]{inputenc}

\usepackage[protrusion=true,expansion=true]{microtype}


\usepackage[style=numeric,
%  backref=true,
 isbn=false,
 maxnames=3,
 maxbibnames=99 ,                
 uniquename=init ,
]{biblatex}
\bibliography{literature.bib}

\usepackage{ownstylesansthm}

\usepackage{comment}
\excludecomment{Long}\includecomment{Short}
% \includecomment{Long}\excludecomment{Short}

\pagestyle{plain}

\begin{document}

\title{Terminal semantics for codata types\\in Homotopy Type Theory}

\author{Benedikt Ahrens and R\'egis Spadotti}

\institute{
Institut de Recherche en Informatique de Toulouse\\
Universit\'e Paul Sabatier, 
Toulouse}

\newcommand{\fat}[1]{\textbf{#1}}





\maketitle

% \tableofcontents

\begin{abstract}

 We extend and adapt previous work on terminal semantics of coinductive data types in intensional Martin-Löf type theory to
 obtain terminal semantics for such data types in a more extensional variant of type theory, the recently developed 
 \emph{Homotopy Type Theory} a.k.a.\ \emph{Univalent Foundations}.
   
\end{abstract}




\section{Introduction}
 
 In previous work \parencite{trimat_coq} we presented a terminal coalgebra semantics for coinductive data types in
 intensional Martin-Löf type theory (IMLTT) \parencite{martin_lof}.
 One of the main characteristics of coinductive data types in IMLTT is that the notions of \emph{equality}
 (as given by the Martin-Löf identity type) and \emph{bisimilarity}---a form of observational equivalence---do \emph{not
 coincide}. Properties of inhabitants of such data types and of maps into such a codata type are hence stated and proved modulo bisimilarity rather than equality.
 
 \emph{Homotopy Type Theory} (HoTT) \parencite{hottbook} is a variant of Martin-Löf type theory which features more extensional 
 characteristics; in particular, set-level quotients exist in Homotopy Type Theory \parencite[Chap.\ 6.10]{hottbook}.
 
 In this work, we profit from the existence of quotients in HoTT and use them to quotient codata types with respect to their
 bisimilarity relation, thus reconciling bisimilarity and equality in the aforementioned sense.
 We then prove a variant of an earlier result \parencite{trimat_coq}: the data type thus obtained constitutes the terminal 
 object in some category.
 The difference between the variant proved in that earlier work and the variant of the present work is that uniqueness of the terminal
 morphisms is proved modulo pointwise \emph{bisimilarity} in the former and pointwise \emph{equality} in the latter.
 
 
\printbibliography


\appendix


% \section{Rules for $\stream$ and bisimilarity}\label{stream_rules}

\paragraph*{Rules for $\stream$}



\begin{description}

 \item[Formation]\hfill \\
\mbox{\hfill} 
 \begin{center}
 \def\extraVskip{3pt}
     \def\proofSkipAmount{\vskip.8ex plus.8ex minus.4ex}

         
   \AxiomC{$A : \Set$}
    \UnaryInfC{$\stream A : \Set$}
     \DisplayProof
 \end{center} 
\mbox{\hfill}
\mbox{\hfill}
 \item[Elimination]\hfill \\
\mbox{\hfill}
\begin{center}
    \AxiomC{$t : \stream A$} %\doubleLine
     \UnaryInfC{$\shead_A~t : A$}
      \DisplayProof
                        \hspace{3ex}
                                       \AxiomC{$t : \stream A$}%\doubleLine
                                       \UnaryInfC{$\stail_A~t : \stream A$}
                                       \DisplayProof%
\end{center}
\mbox{\hfill}
\mbox{\hfill}
  \item[Introduction]\hfill \\                                     
\mbox{\hfill}
  \begin{center}
               \AxiomC{$T : \Set$} \AxiomC{$hd : T \to A$} \AxiomC{$tl : T \to T$} %\doubleLine
               \TrinaryInfC{$\corec_A~hd~tl : T \to \stream A$}
               \DisplayProof%
\end{center}
\mbox{\hfill}
\mbox{\hfill}         
  \item[Computation]\hfill \\
\mbox{\hfill}
\begin{center}          
               \AxiomC{$hd : T \to A$} \AxiomC{$tl : T \to T$} \AxiomC{$t:T$}%\doubleLine
               \TrinaryInfC{$\shead_A(\corec_A~hd~tl~t) = hd(t)$}
               \DisplayProof
               
               \vspace{1em}
               
               \AxiomC{$hd : T \to A$} \AxiomC{$tl : T \to T$} \AxiomC{$t:T$}%\doubleLine
               \TrinaryInfC{$\stail_A(\corec_A~hd~tl~t) = \corec_A~hd~tl~(tl~t)$}
               \DisplayProof
\end{center}

 \end{description}              
               
\paragraph*{Bisimilarity on $\stream$} %\label{stream_bisim}          
 

 
\begin{description}

 \item[Formation]\hfill \\
\mbox{\hfill} 
 \begin{center}
 \def\extraVskip{3pt}
     \def\proofSkipAmount{\vskip.8ex plus.8ex minus.4ex}

         
   \AxiomC{$A : \Set$} \AxiomC{$s, t : \stream A$}
    \BinaryInfC{$\bisim_A~s~t : \Set$}
     \DisplayProof
 \end{center} 
\mbox{\hfill}
\mbox{\hfill}
 \item[Elimination]\hfill \\
 \mbox{\hfill}
\begin{center}
    \AxiomC{$s, t : \stream A$} \AxiomC{$p: \bisim_A~s~t$} %\doubleLine
     \BinaryInfC{$\shead_A~s = \shead_A~t$}
      \DisplayProof
                        \hspace{3ex}
                                       \AxiomC{$s,t : \stream A$} \AxiomC{$p: \bisim_A~s~t$}%\doubleLine
                                       \BinaryInfC{$\bisim_A (\stail_A~s) (\stail_A~t)$}
                                       \DisplayProof%
\end{center}
\mbox{\hfill}
\mbox{\hfill}
  \item[Introduction]\hfill \\                                     
\mbox{\hfill}                        
\begin{center}

\def\fCenter{~\mbox{$\entails$}}
\def\ScoreOverhang{30pt}

               \Axiom$\fCenter\ R : \stream A \to \stream A \to \Set$ \noLine\UnaryInf$x,y : \stream A \fCenter\ R~x~y \to \shead~x = \shead~y$ \noLine
                \UnaryInf$x,y : \stream A \fCenter\ R~x~y \to R~(\stail~x) (\stail~y)$  %\doubleLine
               \UnaryInf$x,y : \stream A \fCenter\ R~x~y \to \bisim~x~y$
               \DisplayProof%
\end{center}
                      
%           \vspace{1em}


 \end{description}              
               
               
               

% \section{Rules for $\Tri$ and bisimilarity}\label{tri_rules}

\paragraph*{Rules for $\Tri$}

\begin{description}

 \item[Formation]\hfill \\
 
 \begin{center}
 \def\extraVskip{3pt}
     \def\proofSkipAmount{\vskip.8ex plus.8ex minus.4ex}

         
   \AxiomC{$A : \Set$}
    \UnaryInfC{$\Tri A : \Set$}
     \DisplayProof
 \end{center} 
 
 \item[Elimination]\hfill \\
 

%   \hspace{3ex}

\begin{center}
    \AxiomC{$t : \Tri A$} %\doubleLine
     \UnaryInfC{$\head_A~t : A$}
      \DisplayProof
                        \hspace{3ex}
                                       \AxiomC{$t : \Tri A$}%\doubleLine
                                       \UnaryInfC{$\tail_A~t : \Tri(E\times A)$}
                                       \DisplayProof%
\end{center}
  \item[Introduction]\hfill \\                                     
                       
%           \vspace{1em}             
            
\begin{center}
               \AxiomC{$T : \Set\to\Set$} 
               \AxiomC{$hd : \forall A,TA \to A$} \AxiomC{$tl : \forall A, T A \to T(E\times A)$} %\doubleLine
               \TrinaryInfC{$\corec_T~hd~tl :  \forall A, T A \to \Tri A$}
               \DisplayProof%
\end{center}
                      
%           \vspace{1em}
  \item[Computation]\hfill \\

\begin{center}          
               \AxiomC{$hd : \forall A,TA \to A$} \AxiomC{$tl : \forall A, T A \to T(E\times A)$} %\doubleLine
               \AxiomC{$t : TA$}
               \TrinaryInfC{$\head_T(\corec_A~hd~tl~t) = hd(t)$}
               \DisplayProof
               
               \vspace{1em}
               
               \AxiomC{$hd : \forall A,TA \to A$} \AxiomC{$tl : \forall A, T A \to T(E\times A)$} %\doubleLine
               \AxiomC{$t : TA$}
               \TrinaryInfC{$\tail_T(\corec_A~hd~tl~t) = \corec_A~hd~tl~(tl~t)$}
               \DisplayProof
\end{center}

 \end{description}              
               
\paragraph*{Bisimilarity for $\Tri$}              
 


               
 \begin{description}

 \item[Formation]\hfill \\
 
 \begin{center}
 \def\extraVskip{3pt}
     \def\proofSkipAmount{\vskip.8ex plus.8ex minus.4ex}

         
   \AxiomC{$A : \Set$} \AxiomC{$s, t : \Tri A$}
    \BinaryInfC{$\bisim_A~s~t : \Set$}
     \DisplayProof
 \end{center} 
 
 \item[Elimination]\hfill \\
 

%   \hspace{3ex}

\begin{center}
    \AxiomC{$s, t : \Tri A$} \AxiomC{$p: \bisim_A~s~t$} %\doubleLine
     \BinaryInfC{$\head_A~s = \head_A~t$}
      \DisplayProof
                        \hspace{3ex}
                                       \AxiomC{$s,t : \Tri A$} \AxiomC{$p: \bisim_A~s~t$}%\doubleLine
                                       \BinaryInfC{$\bisim_A (\tail_A~s) (\tail_A~t)$}
                                       \DisplayProof%
\end{center}
  \item[Introduction]\hfill \\                                     
                       
%           \vspace{1em}             
            
\begin{center}
               \AxiomC{$R : \forall A, \Tri A \to \Tri A \to \Set$} \noLine
               \UnaryInfC{$\forall A,\forall~s,t : \Tri A, R~s~t \to \head~s = \head~t$} \noLine
                \UnaryInfC{$\forall  A,\forall~s,t : \Tri A, R~s~t \to R~(\stail~s) (\stail~t)$}  %\doubleLine
               \UnaryInfC{$\forall A,\forall~s,t : \Tri A, R~s~t \to \bisim~s~t$}
               \DisplayProof%
\end{center}

\end{description}
               
% 
\section{Correspondance of informal and formal definitions}\label{sec:table_formal_informal}

{

\Crefname{definition}{Def.}{Defs.}
\Crefname{theorem}{Thm.}{Thms.}
\Crefname{example}{Ex.}{Exs.}

\begin{center}
{\renewcommand{\arraystretch}{1.2}
\begin{tabular}{lll}
Informal & Reference & Formal \\ \hline
Category &  & \lstinline!Category!\\
Functor &  & \lstinline!Functor!\\
Relative comonad & \Cref{def:rel_comonad} & \lstinline!RelativeComonad!\\
Triangular matrices as comonad & \Cref{ex:tri_comonad} & \lstinline!Tri!\\
Comodule over comonad & \Cref{def:comodule} & \lstinline!Comodule!\\
% Morphism of comodules & \Cref{def:morphism_of_comodules}& \lstinline!Comodule.Morphism!\\
Tautological comodule & \Cref{def:tautological_comodule} &\lstinline!tcomod!\\

Pushforward comodule & \Cref{def:pushforward_comodule} & \lstinline!pushforward!\\
Induced comodule morphism &\Cref{def:induced} & \lstinline!induced_morphism!\\
Relative comonad with cut &\Cref{def:rel_comonad_with_cut} & \lstinline!RelativeComonadWithCut!\\
Precomposition with product & \Cref{def:product_in_context} &\lstinline!precomposition_with_product!\\
$\tail$ is comodule morphism &\Cref{ex:tail_comodule} & \lstinline!Rest!\\
Coalgebras of triangular matrices & \Cref{def:cat_tri} & \lstinline!TriMat!\\
Triangular matrices are terminal & \Cref{ex:final_sem_tri} & \lstinline!Coinitiality!\\
\end{tabular}
}
\end{center}


}


\end{document}



















